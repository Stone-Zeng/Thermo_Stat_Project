\documentclass{article}

\usepackage{geometry}
	\geometry{a4paper, twoside, left = 1.91 cm, right = 1.91 cm, top = 2.54 cm, bottom = 2.54 cm, headheight = 3 cm}

\usepackage{titlesec}

\usepackage{xcolor}
\usepackage[hyperfootnotes=false]{hyperref}
	\hypersetup{
		bookmarksopen = true,
		bookmarksopenlevel = 1,
		bookmarksnumbered = true,
		pdftitle = {基于Heisenberg模型的skyrmion研究},
		pdfauthor = {杜治纬,孙晓晨,陶润恺,王金银,曾祥东},
		colorlinks,
		linkcolor = {red!60!black},
		citecolor = {green!50!black},
		urlcolor = {blue!70!black}
	}

\usepackage[stable, perpage]{footmisc}
\usepackage{pifont}
	\renewcommand{\thefootnote}{\ding{\numexpr171+\value{footnote}}}

\usepackage[numbers, square, super]{natbib}
	\setlength{\bibsep}{0 pt}
%	\renewcommand{\citenumfont}{\itshape}
	\renewcommand{\bibsection}{\section{参考文献}}
	\bibliographystyle{gbt-7714-2015-numerical}

\usepackage{amsmath}
\usepackage{mathtools}
\usepackage[adobe-garamond]{mathdesign}
\usepackage{bm}

\usepackage[no-math]{fontspec}
	\setmainfont[Ligatures = TeX, BoldFont = AGaramondPro-Semibold.otf, ItalicFont = AGaramondPro-Italic.otf]{AGaramondPro-Regular.otf}

\usepackage[UTF8, heading = true, 10pt]{ctex}
\usepackage{xeCJK}
	\xeCJKsetup{
		AutoFakeBold = false,
		AutoFakeSlant = false
	}
	\setCJKmainfont[BoldFont = 华文中宋, ItalicFont = 华文楷体, Mapping = fullwidth-stop]{宋体}
	\setCJKfamilyfont{中宋}[Mapping = fullwidth-stop]{华文中宋}
	\setCJKfamilyfont{楷体}[Mapping = fullwidth-stop]{华文楷体}
	\setCJKfamilyfont{仿宋}[Mapping = fullwidth-stop]{华文仿宋}
	\setCJKfamilyfont{黑体}[BoldFont = 黑体, Mapping = fullwidth-stop]{黑体}
	
	\newfontfamily \courier {Courier New}
	\newfontfamily \verdana {Verdana}
	\newfontfamily \helvetica {HelveticaNeueLTPro-Roman.otf}[BoldFont = HelveticaNeueLTPro-Roman.otf]
	
	\newcommand{\myHeavy}{\helvetica \CJKfamily{黑体}}
	\newcommand{\myItalic}{\normalfont \CJKfamily{楷体}}
	\renewcommand{\sffamily}{\verdana}
	\renewcommand{\ttfamily}{\courier}
	\pagestyle{plain}
	\ctexset{
		appendix = {
			name = {附录}
		},
		section = {
			format = {\myHeavy \Large \centering},
			name = {,、\hspace{-1 em}},
			number = \chinese{section},
		},
		subsection = {
			format = {\myHeavy \large}
		},
	}
	\renewcommand{\abstractname}{\myHeavy \large 摘 \quad 要}

\usepackage{enumitem}

\usepackage{multicol}
\setlength{\columnsep}{20 pt}

\usepackage{listings}
\lstset{
	language = C++,
	basicstyle = \ttfamily \footnotesize,
	breaklines = true,
	tabsize = 4,
	numbers = left,
	numberstyle = \sffamily \itshape \tiny \color{gray},
	numbersep = 10 pt,
	showstringspaces = false,
	keywordstyle = \bfseries \color[rgb]{0,0,1},
	commentstyle = \color[rgb]{0,0.5,0},
	stringstyle = \color[rgb]{0.64,0.082,0.082},
	morecomment = [l][\color{violet}]{\#}
}

\usepackage{siunitx}
	\sisetup{
		number-math-rm = \ensuremath,
		inter-unit-product = \ensuremath{{}\cdot{}},
		group-digits = true,
		group-minimum-digits = 5,
		group-separator = \text{~},
		separate-uncertainty = true,
		range-phrase = \text{~}$\sim$\text{~},
		range-units = single
	}
\usepackage[version = 4]{mhchem}

\usepackage{array}
	\newcolumntype{M}{>{$}c<{$}} %数学模式,居中
\usepackage{tabularx}
	\newcolumntype{Y}{>{\centering\arraybackslash}X} %定宽居中
\usepackage{booktabs} %三线表

\usepackage{graphicx}

\usepackage{float} %浮动体H选项
\usepackage{caption}
	\DeclareCaptionFont{myFigureCaptionFont}{\small}
	\DeclareCaptionFont{myTableCaptionFont}{\myHeavy \small}
	\captionsetup[figure]{
		font = myFigureCaptionFont,
		labelsep = quad,
		skip = 10 pt,
		position = bottom,
		justification = centerfirst,
		format = plain,
		hypcap = false
	}
%	\captionsetup[table]{
%		font = myTableCaptionFont,
%		labelsep = quad,
%		skip = 10 pt,
%		position = top
%	}
%\usepackage[labelformat=simple]{subcaption} %并排图表
%\makeatletter
%\renewcommand\p@subfigure{\thefigure-}
%\renewcommand\thesubfigure{(\alph{subfigure})} %Subfig reference format: x-(x)
%\makeatother

\input{Symbols.tex}

\newcommand{\emphA}[1]{{\myHeavy #1}}
\newcommand{\emphB}[1]{{\myItalic #1}}
\newcommand{\figCaption}[2][]{\vspace{-0.8cm}\captionof{figure}{#2 \\ \emphB{#1}}}

\newenvironment{myAbstract}
	{\begin{abstract} \normalsize}
	{\end{abstract}}

\newenvironment{myFigure}
	{\par\medskip\noindent\minipage{\linewidth}\centering}
	{\endminipage\par\medskip}

\title{
	\vspace{-2 cm} \LARGE \bfseries
	基于Heisenberg模型的skyrmion研究
}
\author{
	\CJKfamily{楷体}
	杜治纬,孙晓晨,陶润恺\thanks{本小组组长。联系方式:\href{mailto:14307130043@fudan.edu.cn}{\tt 14307130043@fudan.edu.cn}。},王金银,曾祥东
}
%杜治纬,孙晓晨,陶润恺\thanks{本小组组长。},王金银,曾祥东

\date{
	\CJKfamily{楷体}
	\today
}

\begin{document}
	\maketitle
	\vspace{-0.6cm}
	\begin{myAbstract}
		我们从铁磁相变的Heisenberg模型出发,利用Metropolis算法模拟了系统各热力学参量随温度的变化,讨论了不同步数下铁磁相的演变。进一步地,我们对加入外场和DM相互作用情形下的铁磁体系拓扑相变进行了研究,画出了各种相的图像,并绘制了相图,还简要讨论了铁磁体中这些相变产生的机制。之后又探讨了体系在向Ising情形和XY情形转变中出现的现象。最后我们通过求解非线性 $\s$ 模型,从理论角度进行了稍许完善。
	\end{myAbstract}
	
	\begin{multicols}{2}
	\section{引言}
		Skyrmion(斯格米子)是由T.~Skyrme在1962提出来的一种拓扑激发态,它是与重子有关的、受拓扑保护的准粒子,包括重子与共振态的叠加。在空间反演对称性破缺的手征磁体中,skyrmion可稳定存在;之后,在各种凝聚态物质体系中也发现了它的存在\cite{yu2010real}。它的尺度通常处在纳米数量级\cite{ulrich2011chiral}。但目前自旋体系中形成稳定skyrmion结构的机制尚不明确,驱动skyrmion运动的机理也有待进一步研究。
		
		在场论中,skyrmion是非线性 $\sigma$ 模型中的一个非平庸经典解,是一种拓扑孤子。在二维磁性系统中,假设平面内各向同性,可以形成如图~\ref{FIG_SKYRMIONS_WIKI} 所示的中心对称结构,此结构的非最低自由能相在特定条件下是可以存在的。
		\begin{myFigure}
			\includegraphics[width = 5 cm]{"Figures/FIG_Skyrmions_WIKI.png"}
			\vspace{0.8cm}
			\figCaption[图 (a) 反映的是hedgehog(刺猬)型skyrmion,图 (b) 反映的是涡旋型skyrmion。]{二维skyrmion的示意图\cite{wiki:MagneticSkyrmion}}
			\label{FIG_SKYRMIONS_WIKI}
		\end{myFigure}
		
		由于skyrmion具有良好的抗干扰特性,并且尺度足够小,因此它有望成为新型信息存储介质。相较于同样有希望成为存储介质的磁畴壁\cite{parkin2008magnetic},驱动skyrmion所需的电流更小\cite{jonietz2010spin}。因此基于skyrmion结构,人们将有机会开发出高密度、低能耗的新型磁存储器件。
		
	\section{Heisenberg模型}
		\subsection{介绍}
			(经典)Heisenberg模型是开展skyrmion等一系列研究的基础,它是 $n$-矢量模型在 $n = 3$ 时的情况。同时,与文献中的结果对比,也可以借此验证我们程序的正确性。
			
			取一个 $d$ 维点阵,每个点阵点均是具有单位长度的三维自旋矢量
			\begin{equation}
				\vecb{s}_i \in \mathbb{R}^3, \quad \abs{\vecb{s}_i} = 1 \fullstop
			\end{equation}
			若系统的Hamilton量定义为
			\begin{equation} \label{EQ_DEF_OF_HEISENBERG_MODEL}
				\scH = -\sum_{\mean{i,j}} J_{i\!j} \vecb{s}_i \cdot \vecb{s}_i - H_z\sum_i s_i^z \comma
			\end{equation}
			则该自旋系统就是Heisenberg模型\cite{binder2010monte,joyce1967classical}。式~\eqref{EQ_DEF_OF_HEISENBERG_MODEL} 中的
			\begin{equation}
				J_{i\!j} =
				\begin{cases}
					J, & \text{若 $i$、$j$ 为近邻} \semicomma \\
					0, & \text{其他情况} \comma
				\end{cases}
			\end{equation}
			而 $H_z$ 是外界磁场的 $z$ 分量。在Hamilton量中添加其他的相互作用,就可以得到各种外场下的情况。
			
		\subsection{算法}
			我们首先研究二维Heisenberg模型,并且不考虑外场。利用Metropolis算法(一种Monte Carlo算法),可以对该模型进行模拟。该算法的思路如下:
			\begin{enumerate}[itemsep = 0 pt, parsep = 0 pt, topsep = 0 pt]
				\item 给出自旋矢量在平面点阵中的一个随机分布。
				\item 选择一个自旋 $\vecb{s}_i$。
				\item 随机生成一个自旋矢量 $\vecb{s}'_i$,计算从 $\vecb{s}_i$ 翻转到 $\vecb{s}'_i$ 所带来的的能量变化 $\incr E$,从而得到翻转概率
					\begin{equation}
						P =
						\begin{cases}
							\exp \left(-\dfrac{\incr E}{\kB T}\right), &\incr E > 0 \semicomma \\
							1, &\incr E \leqslant 0 \comma
						\end{cases}
					\end{equation}
				\item 生成一个 $[0,1)$ 之间的随机数。若该随机数小于翻转概率 $P$,则将 $\vecb{s}_i$ 翻转为 $\vecb{s}'_i$。
				\item 重复进行步骤 $2\sim4$。我们采取的方式是遍历整个点阵,将它记为一个Monte Carlo步(MCS)。进行若干MCS后,体系就可以达到稳态。
				\item 在最后的若干MCS中作系综平均,计算各物理量。\cite{周琼2010蒙特卡洛方法在磁性系统中的应用}
			\end{enumerate}
			
			在我们的模拟中,整个过程包含一定数量的MCS,并在最后50步中统计能量和磁矩。另外,所有参数均采用约化单位。其中,取 $\kB=1$,$J=1$。
			
			下面给出各物理量的计算公式。总能量
			\begin{equation}
				E = \sum_{i} \ve_i \comma
			\end{equation}
			这里的 $\ve_i$ 即单个格点的Hamilton量。总磁矩
			\begin{equation}
				\vecb{M} = \sum_{i} \vecb{s}_i \fullstop
			\end{equation}
			比热\cite{林宗涵2007热力学与统计物理学}
			\begin{equation} \label{EQ_HEAT_CAPACITY}
				C = \frac{\mean{E^2} - \mean{E}^2}{\kB T^2}
				= \frac{1}{\kB T^2} \left[ \frac{\sum_{i} \ve_i^2}{L^2} - \left( \frac{\sum_{i} \ve_i}{L^2} \right)^2 \right] \semicomma
			\end{equation}
			磁化率
			\begin{equation} \label{EQ_MAGNETIC_SUSCEPTIBILITY}
				\c = \frac{\mean{\vecb{M}^2} - \mean{\vecb{M}}^2}{\kB T}
				= \frac{1}{\kB T} \left[ \frac{\sum_{i} \vecb{s}_i^2}{L^2} - \left( \frac{\sum_{i} \vecb{s}_i}{L^2} \right)^2 \right] \fullstop
			\end{equation}
			以上两式中的 $L$ 是点阵边长,$\kB$ 是Boltzmann常数。
			
		\subsection{结果与分析}
			首先取 $36\times36$ 的点阵,进行1000MCS。温度 $T$ 从0改变到4,记录平均能量 $\ve$ 和平均磁矩 $\vecb{m}$ 随温度的变化\footnote{
				由于点阵大小有变,因此这里作了归一化,以方便比较。
			},如图~\ref{FIG_HEISENBERG_E_M_VS_T_LATTICE_36}。
			
			与文献中给出的结果\cite{周琼2010蒙特卡洛方法在磁性系统中的应用,watson1970classical}比较,发现能量曲线符合得较好,而磁矩曲线差别较大。主要问题在于低温部分。从图~\ref{FIG_HEISENBERG_E_M_VS_T_LATTICE_36} 可以看出,低温部分有相当严重的涨落。这是点阵尺寸过大及模拟步数不够导致的。(见附录~\ref{SEC_演化情况})
			
			因此,我们把点阵大小调整为 $10\times10$,同时将MCS增加到 \num{100000}。这时得到的结果就与文献中给出的结果符合得很好,磁矩在低温下的涨落也有所减弱,如图~\ref{FIG_HEISENBERG_E_M_VS_T_LATTICE_10}。这就验证了我们的程序的正确性。
			
			\begin{myFigure}
				\includegraphics[width = \textwidth]{Figures/FIG_HEISENBERG_E_M_VS_T_LATTICE_36}
				\figCaption[点阵大小为 $36\times36$,步数为1000。灰色曲线表示原始数据,蓝色曲线表示平滑后的结果;数据已进行单位化,即将原始值除以格点数。]{平均能量、平均磁矩大小随温度的变化}
				\label{FIG_HEISENBERG_E_M_VS_T_LATTICE_36}
			\end{myFigure}
			
			\begin{myFigure}
				\includegraphics[width = \textwidth]{Figures/FIG_HEISENBERG_E_M_VS_T_LATTICE_10}
				\figCaption[点阵大小为 $10\times10$,步数为 \num{100000}。]{平均能量、平均磁矩大小随温度的变化}
				\label{FIG_HEISENBERG_E_M_VS_T_LATTICE_10}
			\end{myFigure}
			\begin{myFigure}
				\includegraphics[width = \textwidth]{Figures/FIG_HEISENBERG_C_X_VS_T_LATTICE_10}
				\figCaption[彩色曲线的数据由式~\eqref{EQ_HEAT_CAPACITY} 和式~\eqref{EQ_MAGNETIC_SUSCEPTIBILITY} 求得;而左图中的灰色曲线是利用能量曲线导数(差分)计算出的结果。]{热容、磁化率随温度的变化}
				\label{FIG_HEISENBERG_C_X_VS_T_LATTICE_10}
			\end{myFigure}
			
			按照该条件下的结果绘制热容与磁化率曲线,如图~\eqref{FIG_HEISENBERG_C_X_VS_T_LATTICE_10}。利用式~\eqref{EQ_HEAT_CAPACITY} 和式~\eqref{EQ_MAGNETIC_SUSCEPTIBILITY} 得到的结果在低温下是发散的,这是因为Heisenberg对应经典的连续情况,而在低温下必须考虑量子效应。对于热容曲线,我们还用之前能量曲线的导数来进行计算,也在图~\ref{FIG_HEISENBERG_C_X_VS_T_LATTICE_10} 中。这实际上对应热容的定义,是不发散的合理结果。
			
			\begin{myFigure}
				\includegraphics[width = \textwidth]{Figures/FIG_HEISENBERG_E_M_VS_T_LATTICE_10_AFM}
				\figCaption[点阵大小为 $10\times10$,步数为 \num{100000}。]{平均能量、平均磁矩大小随温度的变化(反铁磁)}
				\label{FIG_HEISENBERG_E_M_VS_T_LATTICE_10_AFM}
			\end{myFigure}
			
			之后模拟反铁磁相互作用(AFM)下的情况。取 $J=-1$,其余参数同上。如图~\ref{FIG_HEISENBERG_E_M_VS_T_LATTICE_10_AFM} ,能量曲线与铁磁相互作用下几乎完全一样,而平均磁矩却始终为零。原因在于AFM使得邻近的磁矩存在反向的趋势,因而即使是低温,平均磁矩也是零。而在高温下,熵更占优势,自然磁矩处于混乱、无序状态,平均磁矩也为零。
			
			Heisenberg模型的相变是能量与熵竞争的结果。低温下,能量更占优势。因此相互作用的影响显著。$J>0$,即出现铁磁相;$J<0$,则出现反铁磁相。高温下,熵的作用超过能量。此时体系的状态就与 $J$ 的取值无关,都变现为无序与混乱。能量与熵的影响达到平衡时,即为临界点。
			
			考察体系状态随步数的演化过程,可以有更多的发现。在低温下,无论是FM还是AFM,均需要相当多的步数才可以达到稳定的有序态。即使是在有序态,也并非处于磁矩取向完全相同的理想情况(以FM为例),而是大片取向相同的花斑,如图~\ref{FIG_HEISENBERG_STEP_FM}。由于Metropolis算法是逐点翻转,当出现大范围取向相同的区域时,翻转任意一个自旋所需的能量就会很大,因此难以进行。此即所谓“临界慢化”现象,需要更高级的算法来克服。
			
			\begin{myFigure}
				\includegraphics[width = \textwidth]{Figures/FIG_HEISENBERG_STEP_FM}
%				\vspace{0.5cm}
				\figCaption[彩色矩阵图反映的是自旋 $z$ 分量的大小,矢量图反映的是自旋在 $xy$ 平面上的投影。从左至右依次为1000、\num{10000}、\num{100000} MCS。]{铁磁相随模拟步数的演化}
				\label{FIG_HEISENBERG_STEP_FM}
			\end{myFigure}
			
			对于AFM,在较小的点阵中临界慢化不很明显。如图~\ref{FIG_HEISENBERG_STEP_AFM},并没有出现大片的花斑,而是几乎严格的理论情况,即相邻各点去向均相反。不过,在较大的点阵中,仍然是有花斑出现的。
			
			\begin{myFigure}
				\includegraphics[width = \textwidth]{Figures/FIG_HEISENBERG_STEP_AFM}
				\figCaption[从左至右依次为1000、\num{10000}、\num{100000} MCS。]{反铁磁相随模拟步数的演化}
				\label{FIG_HEISENBERG_STEP_AFM}
			\end{myFigure}
			
			体系状态随步数的演化还与点阵的大小有关,稍微详细的讨论见附录~\ref{SEC_演化情况}。
			
	\section{DM相互作用与skyrmion}
		\subsection{介绍}
			在非中心对称铁磁体中,存在一种特殊的相互作用——手征相互作用,它使得skyrmion这种拓扑结构可以作为一种激发态而存在。手征相互作用主要存在如下四种机制\cite{代杰2015手征磁体中的}:
			\begin{enumerate}[itemsep = 0 pt, parsep = 0 pt, topsep = 0 pt]
				\item  非中心对称的磁性材料中,自旋轨道耦合作用的高阶项,即Dzyaloshinsky--Moriya(DM)相互作用\cite{dzyaloshinsky1958thermodynamic,moriya1960anisotropic},这是一种反对称交换作用;
				\item 四自旋交换相互作用;
				\item 长程的磁偶极--偶极相互作用;
				\item 阻挫交换作用。
			\end{enumerate}
			
			我们的讨论和模拟仅限于DM相互作用导致的skyrmion。DM相互作用最初由Dzyaloshinsky在1958年提出。他认为在对称度不够高的情形下,自旋方向会倾斜,而并非保持平行。\cite{dzyaloshinsky1958thermodynamic} 1960年,T.~Moriya利用Anderson超交换作用的微扰方法对DM相互作用做了进一步阐释,并给与了微观的理论依据。\cite{moriya1960anisotropic}
			
			\begin{myFigure}
				\includegraphics[width = 7 cm]{Figures/FIG_DM_INTERACTION}
				\vspace{0.8 cm}
				\figCaption[图 (a) 反映的是由两个原子自旋和一个带有强SOC的原子造成的DM相互作用,图 (b) 反映的是铁磁相金属(灰色)和有强SOC金属交界处的DM相互作用。]{DM相互作用\cite{fert2013skyrmions}}
				\label{FIG_DM_INTERACTION}
			\end{myFigure}
			
			如图~\ref{FIG_DM_INTERACTION},DM相互作用由界面附近空间对称性的破坏而产生,其对应的Hamilton量为:
			\begin{equation}
				\scH_\text{DM} = \vecb{D}_{12} \cdot (\vecb{s}_1 \times \vecb{s}_2) \comma
			\end{equation}
			其中的 $\vecb{s}_1$、$\vecb{s}_2$ 分别表示两个自旋矢量,而 $\vecb{D}_{12}$ 为DM相互作用参量,它由界面材料和晶体结构决定,是一个与时间无关并且空间非均匀的矢量。$\vecb{D}_{12}$ 具有反对称性,即
			\begin{equation}
				\vecb{D}_{12}=-\vecb{D}_{21} \fullstop
			\end{equation}
			DM相互作用是相邻原子通过与异相中原子的自旋--轨道耦合作用(SOC)发生自旋偏转的原因。自旋-轨道耦合作用愈强烈,$\vecb{D}_{12}$ 矢量的模长越大,自旋方向翻转 \SI{180}{\degree} 时经过的空间长度越短,形成的skyrmion间距就越小。DM相互作用较强的二维磁性材料比材料对称性较强的三维材料在 $T$-$H$ 相图中对应区域更大,前者更易形成skyrmion相。
			
			由DM相互作用而导致出现skyrmion的材料,在实验中发现有 \ce{MnSi}、$\ce{Fe_$1-x$Co_$x$Si}$ 和 $\ce{Mn_$1-x$Fe_$x$Ge}$ 等。这些材料中,skyrmion的尺寸由DM相互作用和铁磁交换作用的强度共同决定,一般在 \SIrange{5}{10}{\nano\meter},要比晶胞尺寸大很多,因此可以认为其满足连续性近似条件(当然在我们的模拟中仍然是离散的)。\cite{代杰2015手征磁体中的}
			
			在连续性近似条件下,自旋铁磁系统的Hamilton量为\cite{nagaosa2013topological}
			\begin{equation}
				\scH = \int \dd r \left[ J(\nabla \vecb{n})^2 + D \vecb{n} \cdot (\nabla \times \vecb{n}) - \vecb{H} \cdot \vecb{n} \right] \comma
			\end{equation}
			式中的 $J$ 是交换作用常数($J>0$ 时为铁磁相互作用),$D$ 是DM相互作用常数,$\vecb{H}$ 是外加磁场。将其离散化,可得\cite{wiki:MagneticSkyrmion}
			\begin{equation}
				\begin{aligned}
					\scH=&-J\sum_{\vecb{r}}\vecb{s}_{\vecb{r}} \cdot
					\left(\vecb{s}_{\vecb{r}\pm\vecb{e}_x}\!+\vecb{s}_{\vecb{r}\pm\vecb{e}_y} \right) \\
					&-D\sum_{\vecb{r}} \left[\vecb{s}_{\vecb{r}}\times\vecb{s}_{\vecb{r}\pm\vecb{e}_x}\!\!\cdot\left(\pm\vecb{e}_x\right) + \vecb{s}_{\vecb{r}}\times\vecb{s}_{\vecb{r}\pm\vecb{e}_y}\!\!\cdot\left(\pm\vecb{e}_y\right) \right] \\
					&-\vecb{H}\cdot\sum_{\vecb{r}}\vecb{s}_{\vecb{r}} \fullstop
				\end{aligned}
			\end{equation}
			该式的第一项为交换作用,也就是式~\eqref{EQ_DEF_OF_HEISENBERG_MODEL} 的第一项;第二、第三项分别为DM相互作用和外加磁场的影响\footnote{
				这里认为三种相互作用都是均匀的。
			}。
			
			这三种相互作用的竞争,将会导致不同的相。在Curie温度以下,当外加磁场较小或为0时,会出现螺旋相(helical phase),如图~\ref{FIG_HELICAL} 中的 (a)。此时Zeeman能忽略不计,只有交换作用与DM相互作用竞争。
			
			若稍稍加大外加磁场,但仍使其低于低于某临界值,则会出现锥形相(conical phase),如图~\ref{FIG_HELICAL} 中的 (b) 和 (c)。此时外加磁场的存在可以改变螺旋的简并度,使自旋倾向与沿外磁场方向排列,但是Zeeman能仍然较小,不足与前两项竞争。
			
			当外加磁场高于该临界值,但不是非常大时,则会出skyrmion相(见图~\ref{FIG_SKYRMIONS_WIKI}),此时发生的是Zeeman能、交换作用与DM相互作用三者间的竞争。对于一个孤立的skyrmion,在其中心自旋向下,反平行于外磁场;而在边缘处,自旋向上,与外磁场平行,中间的自旋则在二维平面内旋转。当磁场非常大的时候,便达到铁磁相(FM),此时Zeeman能非常大,交换作用与DM相互作用都不足以与其竞争。
			
			在温度超过Curie点时,体系将处于无序相,因为相互作用带来的能量均不足以与熵竞争,而微观上的拓扑性质都将被抹平。
			
			\begin{myFigure}
				\includegraphics[width = \textwidth]{"Figures/FIG_Helical.png"}
				\vspace{0 cm}
				\figCaption[图 (a) 的螺旋相中,自旋在垂直于波矢的自旋面内旋转,黑色箭头代表波矢的传播方向;图 (b) 的在锥形相中,外磁场平行于波矢方向,自旋绕着波矢进动,故在外磁场方向上存在一个均一的磁化分量;图 (c) 中,外磁场垂直于波矢传播方向,此时螺旋(helice)变成横向扭曲的螺旋曲面(helicoid)结构。]{螺旋相与锥形相\cite{ulrich2011chiral}}
				\label{FIG_HELICAL}
			\end{myFigure}
			
			为了定量描述skyrmion相的特性,除了能量与磁矩,还需引进skyrmion密度(SkD)\cite{muhlbauer2009skyrmion}
			\begin{equation} \label{EQ_SKD_CONTINUOUS}
				\f=\frac{1}{4\p}\vecb{s}\cdot \left(\frac{\pd \vecb{s}}{\pd x}\times\frac{\pd \vecb{s}}{\pd y}\right) \comma
			\end{equation}
			其中的 $x$、$y$ 是垂直于磁场 $\vecb{H}$ 的坐标分量。skyrmion密度实际就是下文将引入的拓扑不变量。当 $\abs{\f}$ 接近1时,表明出现了稳定的拓扑结构,即skyrmion;而其余各种相,均有 $\f\approx 0$。将其离散化,并忽略系数,可得
			\begin{equation}
				\f=\sum_{\vecb{r}} \vecb{s}\cdot\left[ \left(\vecb{s}_{\vecb{r}+\vecb{e}_x}\!-\vecb{s}_{\vecb{r}-\vecb{e}_x}\right) \times\left(\vecb{s}_{\vecb{r}+\vecb{e}_y}\!-\vecb{s}_{\vecb{r}-\vecb{e}_y}\right)\right] \fullstop
			\end{equation}
			这是对所有自旋的求和,即式~\eqref{EQ_SKD_CONTINUOUS} 在整个区域上的积分。
			
			利用Monte Carlo算法模拟skyrmion的思路与方法,和前文所述的Heisenberg模型完全相同,只需改变翻转概率中 $\incr E$ 的计算。模拟程序也是基本一致的。
			
		\subsection{结果与分析}
			首先,仍是利用 $36\times36$ 的点阵,进行1000MCS。取 $J=1$、$D=2.5$、$T=0.1$,外磁场 $H$ 从0改变到5。记录skyrmion密度 $\f$ 和磁矩 $z$ 分量 $m_z$ 随温度 $T$ 的变化,如图~\ref{FIG_PHASE_CURVE} 中的蓝色曲线所示。
			
			可以看出,skyrmion相(SkD在 $H\simeq2$ 处的峰值)和铁磁相($H\gtrsim4$ 后,$\f=0$,且$m_z=1$)是非常明显的,但锥形相的形成受到了skyrmion的抑制,表现在矢量图中即螺旋与skyrmion的混合,如图~\ref{FIG_ORIGINAL_VECTOR_CONICAL}。
			\begin{myFigure}
				\includegraphics[width = \textwidth]{Figures/FIG_PHASE_CURVE}
				\figCaption{Skyrmion密度和磁矩 $z$ 分量随外加磁场的变化}
				\label{FIG_PHASE_CURVE}
			\end{myFigure}
			\begin{myFigure}
				\includegraphics[width = 3 cm]{Figures/FIG_ORIGINAL_VECTOR_CONICAL}
				\vspace{0.8cm}
				\figCaption[步数为1000,其余参数见上文。]{受skyrmion抑制的锥形相(磁矩在 $xy$ 平面上的投影)}
				\label{FIG_ORIGINAL_VECTOR_CONICAL}
			\end{myFigure}
			
			由于skyrmion的拓扑稳定性,它一旦出现就难以消失。与螺旋相或锥形相共存时,skyrmion会阻断螺旋管的形成,使其发生转向,形成类似电路板的结构。
			
			根据之前模拟Heisenberg模型的经验,我们增加了模拟的步数。\num{10000} 和 \num{100000} 步下的结果都绘制在图~\ref{FIG_PHASE_CURVE} 中。此时可以看出,随着步数的增加,锥形相逐渐变得明显。三种相\footnote{
				螺旋相与锥形相的图像是类似的,但在我们的模拟中,螺旋相出现的条件比较苛刻。我们得到的结果是锥形相。
			}的典型图像见图~\ref{FIG_LATTICE_VECTOR} 、\ref{FIG_LATTICE_MATRICX},以及放在附录~\ref{SEC_额外的图片} 中的图~\ref{FIG_VECTOR_3D}。
			
			\begin{myFigure}
				\includegraphics[width = \textwidth]{Figures/FIG_LATTICE_VECTOR}
				\figCaption{磁矩在 $xy$ 平面上的投影}
				\label{FIG_LATTICE_VECTOR}
			\end{myFigure}
			\begin{myFigure}
				\includegraphics[width = \textwidth]{Figures/FIG_LATTICE_MATRICX}
				\figCaption[锥形相:$J=1\comma B=0\comma D=2.5\comma T=0.2\semicomma$\\
				skyrmion相:$J=1\comma B=2\comma D=2.5\comma T=0.1\semicomma$\\
				铁磁相:$J=1\comma B=4\comma D=2.5\comma T=0.1\fullstop$\\
				点阵大小为 $36\times36$,步数为 \num{100000}。图~\ref{FIG_VECTOR_3D} 中所用的参数与这里相同。]{磁矩的 $z$ 分量图}
				\label{FIG_LATTICE_MATRICX}
			\end{myFigure}
			
			重新观察图~\ref{FIG_PHASE_CURVE}。在 \num{100000} 步的情形下,SkD与 $m_z$ 均存在突跃,这对应着类似二级相变的过程。
			
			在磁矩与SkD随外加磁场变化关系的基础上,我们改变温度,以此绘制相图。点阵大小与MCS仍保持上文的取值,温度 $T$ 从0取到2,而外磁场 $\vecb{H}$ 从 $-4$ 取到4。这里还增加了磁场反向的影响。结果见图~\ref{FIG_PHASE_SD_AND_MZ}。磁场反向后,skyrmion密度也变号,说明skyrmion的涡旋方向也随之改变。而 $m_z$ 分量与外磁场方向、大小的改变均一致,这是预料之中的。
			
			把图~\ref{FIG_PHASE_SD_AND_MZ} 中的两部分合并,就得到了 $T$-$H$ 相图,即图~\ref{FIG_PHASE}。
			
			\begin{myFigure}
				\includegraphics[width = \textwidth]{Figures/FIG_PHASE_SD_AND_MZ}
				\figCaption[步数为 \num{10000},$T$ 轴每精度为 \num{0.05},$H$ 轴精度为 \num{0.08}。]{Skyrmion密度与磁矩 $z$ 方向分量随外加磁场 $\vecb{H}$ 和温度 $T$ 的变化}
				\label{FIG_PHASE_SD_AND_MZ}
			\end{myFigure}
			
			从相图可以看出,随着温度的增大,相的边界逐渐模糊,螺旋相(或锥形相)和skyrmion区域变窄。这说明温度的增加会破坏拓扑结构的存在,使各相的差异逐渐减小。这类似气液相变在临界点之外的情形。锥形相和skyrmion相时刻处在竞争关系中。$T\simeq 0.5$ 时,锥形相的宽度最大,而左右都是skyrmion相主导。我们认为,随着温度的增加,skyrmion更容易形成,从而抑制锥形相的出现。低温下,则是由于Monte Carlo步数的不足,导致螺旋相无法形成。
			
			\begin{myFigure}
				\includegraphics[width = 7cm]{Figures/FIG_PHASE}
				\vspace{0.5cm}
				\figCaption[该图的绘制方法是将SkD和 $m_z$ (实际为其绝对值)用RGB颜色空间的R分量(红色)和G分量(绿色)表示,而B分量(蓝色)取1。]{$T$-$H$ 相图}
				\label{FIG_PHASE}
			\end{myFigure}
			
		\subsection{推广}
			以下我们简单探讨自旋矢量各向异性的影响。定义各向异性指数 $\alpha$ 为自旋 $z$ 分量与 $x$ 或 $y$ 分量大小之比的平均值。$\a=1$ 时,系统为Heisenberg模型;$\a\approach\infty$ 时,退化为为Ising模型(图~\ref{FIG_ORIGINAL_ISING},$\a=100$);$\a\approach 0$ 时,退化为为XY模型(图~\ref{FIG_ORIGINAL_XY},$\a=1/100$)。早已知道,经典Ising模型和XY模型中不存在skyrmion。三维矢量图见附录~\ref{SEC_额外的图片} 中的图~\ref{FIG_ISING_XY_VECTOR_3D}。
			
			\begin{myFigure}
				\includegraphics[width = 7cm]{Figures/FIG_ORIGINAL_ISING}
				\vspace{0.5cm}
				\figCaption[点阵大小为 $36\times36$,步数为 \num{100000},$J=1$,$T=0$。图~\ref{FIG_ORIGINAL_XY} 的参数与此相同。SkD为 \num{-9.72e-2},说明不存在skyrmion。]{只有交换作用的类Ising情形}
				\label{FIG_ORIGINAL_ISING}
			\end{myFigure}
			\begin{myFigure}
				\includegraphics[width = 7cm]{Figures/FIG_ORIGINAL_XY}
				\vspace{0.5cm}
				\figCaption[注意颜色标度与图~\ref{FIG_ORIGINAL_ISING} 中的不同。SkD为 \num{-5.49e-3},说明也不存在skyrmion。]{只有交换作用的类XY情形}
				\label{FIG_ORIGINAL_XY}
			\end{myFigure}
			
			之后改变 $\alpha$ 的值,考察对skyrmion密度的影响。以下模拟中,取 $J=1$,$D=2.5$,$T=0.1$,而 $H$ 从0取到5。共进行 \num{100000} MCS。结果如图~\ref{FIG_LIMIT_ISING} 和图~\ref{FIG_LIMIT_XY}。
			
			\begin{myFigure}
				\includegraphics[width = 5cm]{Figures/FIG_LIMIT_ISING}
				\vspace{0.5cm}
				\figCaption[灰色曲线为原始数据,彩色曲线平滑数据后的结果。图~\ref{FIG_LIMIT_XY} 与此相同。]{趋向Ising极限的过程}
				\label{FIG_LIMIT_ISING}
			\end{myFigure}
			\begin{myFigure}
				\includegraphics[width = 5cm]{Figures/FIG_LIMIT_XY}
				\vspace{0.5cm}
				\figCaption[注意 $\alpha=1/100$ 对应的曲线与水平轴重合。]{趋向XY极限的过程}
				\label{FIG_LIMIT_XY}
			\end{myFigure}
			
			当体系从Heisenberg模型向Ising模型转变时,skyrmion存在的区域不断减小,其强度(SkD)也不断减弱。当完全转变为Ising模型时,便不再出现拓扑结构,因为此时自旋均处于平行或反平行状态,DM相互作用实际上已不存在。
			
			而当体系向XY模型转变时,skyrmion的存在区不断扩大,不过强度也在减弱。对于XY模型,$\vecb{s}_{\vecb{r}}\times\vecb{s}_{\vecb{r}\pm\vecb{e}_x}$ 或 $\vecb{s}_{\vecb{r}}\times\vecb{s}_{\vecb{r}\pm\vecb{e}_y}$ 垂直于 $xy$ 平面,再与 $\vecb{e}_x$ 或 $\vecb{e}_y$ 做点乘后结果为0。因此,DM相互作用也是不存在的,也就不会出现skyrmion。
			
			接着仿照上文的思路,改变外磁场,观察各相的转变,结果分别如图~\ref{FIG_LIMIT_ISING_MATRIX} 和 \ref{FIG_LIMIT_XY_MATRIX} 所示。
			
			\begin{myFigure}
				\includegraphics[width = \textwidth]{Figures/FIG_LIMIT_ISING_VECTOR}
%				\figCaption{Ising演化}
%				\label{FIG_LIMIT_ISING_VECTOR}
			\end{myFigure}
			\begin{myFigure}
				\includegraphics[width = \textwidth]{Figures/FIG_LIMIT_ISING_MATRIX}
				\figCaption[各向异性指数 $\a=100$。从左向右依次为 $B=0$、$B=0.5$、$B=1$ 的结果。其余参数:$J=1$,$D=2.5$,$T=0$。进行 \num{100000} MCS。SkD分别为 \num{7.66e-2}、\num{2.98e-2}、\num{-1.51e-18}。]{类Ising情形下的三种相}
				\label{FIG_LIMIT_ISING_MATRIX}
			\end{myFigure}
			
			\begin{myFigure}
				\includegraphics[width = \textwidth]{Figures/FIG_LIMIT_XY_VECTOR}
%				\figCaption{Ising演化}
%				\label{FIG_LIMIT_XY_VECTOR}
			\end{myFigure}
			\begin{myFigure}
				\includegraphics[width = \textwidth]{Figures/FIG_LIMIT_XY_MATRIX}
				\figCaption[各向异性指数 $\a=1/10$。从左向右依次为 $B=0$、$B=3$、$B=10$ 的结果。其余参数同上。SkD分别为 \num{3.33e-2}、\num{0.649}、\num{3.20e-6}。]{类XY情形下的三种相}
				\label{FIG_LIMIT_XY_MATRIX}
			\end{myFigure}
			
			这里的结果与图~\ref{FIG_LATTICE_VECTOR} 、\ref{FIG_LATTICE_MATRICX} 有一定的类似之处。但在“skyrmion”相上的表现,三者有所不同。在Heisenberg模型中,skyrmion的排列相当整齐,而在XY模型中,其排列几乎没有规律可言;对于Ising模型,甚至连skyrmion的形状也是不规整的。
			
			Ising模型中,由于 $z$ 分量相对较大,交换作用占主导,因此会出现大面积的“磁畴”。加入DM相互作用后,实际影响的只是磁畴的边界。反映在图中是一个带有明确拓扑定向的回路。XY模型中,大范围的磁畴不易出现,这与Heisenberg模型较为接近,因此出现的skyrmion也是类似的小圆环结构。另外,由于 $z$ 分量较小,所以在向铁磁相转变的过程中,突跃远不及类Ising情形,后者的突跃达到了16个数量级。
			
			图~\ref{FIG_LIMIT_XY_MATRIX} 中的 $\a$ 仅取到了 $1/10$。若要与Ising情形相对应而取到 $1/100$,则将无法观测到skyrmion的出现。事实上,在图~\ref{FIG_LIMIT_XY} 中,$\a=1/100$ 对应的曲线完全与水平轴重合;但在图~\ref{FIG_LIMIT_ISING} 中,$\a=100$ 对应的曲线仍然在 $H\simeq 0.2$ 处有一个小峰。这是由于我们研究的体系均是二维平面,在Ising模型中,自旋方向与该平面垂直;而XY模型中,自旋方向总处于该平面内。这其实就使得XY模型成为了一种“双二维”系统\cite{于禄2005边缘奇迹}。对于这种系统,占主导的拓扑结构是类似“源”和“汇”带来的涡旋(见图~\ref{FIG_ORIGINAL_ISING}),而并非skyrmion。
			
	\section{简要的数学}
		\subsection{非线性\texorpdfstring{$\sigma$}{σ}模型}
			非线性 $\sigma$ 模型是一种场论模型,其相互作用是靠流形度规所决定的联络与曲率引入的。不妨讨论 $d$ 维时空中取值在 $n$ 维流形 $M$ 上的场 $u(x)$,流形 $M$ 上有Riemann度规 $g_{i\!j}$。由该度规引入的作用量泛函可以写成\cite{侯伯元2004物理学家用微分几何}
			\begin{equation} \label{EQ_ACTION_FUNCTIONAL}
				I= \frac{1}{2} \int \dd^{\;d} \!\!x g_{i\!j}\pd_\mu \!u^i(x)\pd^{\,\mu} \!\!u^j(x) \fullstop
			\end{equation}
			对作用量求变分即可得非线性 $\sigma$ 模型的解。如果在无穷远点场的曲率有限,$u(x)$ 场可看成是 $S^d \rightarrow M$ 的映射。场 $u(x)$ 可用同伦群 $\pi_d(M)$ 的同伦等价类标志,当 $\pi_d(M)\neq 0$ 时,可能存在拓扑非平庸解。
			
			对应于经典Heisenberg模型,有 $d=2$。此时底流形可紧致为 $S^2$,映射为 $S^2 \rightarrow S^2$,式~\eqref{EQ_ACTION_FUNCTIONAL} 可对应普通铁磁激发。此时由于对应场流形为 $S^2$,$\pi_2 (S_2)$ 即可描述该体系的拓扑不变量\cite{侯伯元2004物理学家用微分几何}
			\begin{equation} \label{EQ_TOPOLOGICAL_INVARIANT}
				\pi_2 (S_2)= \frac{1}{8\p} \int \dd^{\;2} \!x \vecb{N} \cdot \left(\pd_\mu \!\vecb{N} \times \pd_\nu \!\vecb{N}\right) \e^{\mu\nu} \comma
			\end{equation}
			其中的 $\vecb{N}$ 为底流形 $S^2$ 上每一点的单位矢量。该式为丛 $S^2$ 的卷绕数(winding number),是该自旋场底流形上的第一陈数。它是一个拓扑不变量。推导过程见附录~\ref{SEC_拓扑不变量的证明}。
			
		\subsection{二维skyrmion模型}
			在某些缺少反演对称的铁磁体系,它们的铁磁相互作用可以用铁磁交换参数 $J$ 和DM耦合来描述。与非线性 $\sigma$ 模型类似,可以用一个作用量泛函来求解这个体系,这里引入Ginzburg--Landau泛函\cite{臧佳栋2012凝聚态物理学中的拓扑现象}
			\begin{equation}
				F[\vecb{N}]= \frac{J}{2} \sum_{\mu} \left(\pd_\mu \!\vecb{N}\right) \cdot \left(\pd_\mu \!\vecb{N}\right)+D\vecb{N} \cdot \nabla\times\vecb{N} - \vecb{B}\cdot\vecb{N} \comma
			\end{equation}
			式中的第一项是 \eqref{EQ_ACTION_FUNCTIONAL} 式的离散化处理,描述普通铁磁体激发,第二项描述DM相互作用对应的连续模型,第三项则是外磁场贡献。对该体系的作用量做变分处理后,即可求解模型。上文已经通过Metropolis算法模拟了能量最低情况的体系构型。
			
			前文提到的skyrmion指数由式~\eqref{EQ_TOPOLOGICAL_INVARIANT} 决定。拓扑指数不依赖于相互作用形式,可表征该场拓扑非平庸的程度。在计算模拟中即为单位体积内形成的skyrmion数量。
			
		\subsection{螺旋相的数学解释}
			因为 $\mathbb{C}P_1$ 群同构于 $S^2$ 流形,所以自旋矢量可以由 $\mathbb{C}P_1$ 的两个复坐标表示:
			\begin{equation}
				\vecb{Z} =
				\begin{pmatrix}
					z_1 \\ z_2
				\end{pmatrix}
				=
				\begin{pmatrix}
					\ee^{-\ii \vf}\cos \dfrac{\th}{2}\\
					\sin \dfrac{\th}{2}
				\end{pmatrix}
				\comma
			\end{equation}
			它可以通过映射
			\begin{equation}
				\vecb{n}=\vecb{Z}^\dag\vecb{\s}\vecb{Z}
			\end{equation}
			反变换为自旋矢量,其中的 $\vecb{\s}$ 为自旋向量。
			
			在复场表示下重新写出Ginzburg--Landau泛函:
			\begin{equation} \label{EQ_GL_FUNCTIONAL_2}
				\left\{
					\begin{aligned}
						F[\vecb{Z}] &=2J \sum_{\mu} \left(\DD_\mu \!\vecb{Z}\right)^\dag  \left(\DD_\mu \!\vecb{Z}\right) - \vecb{B}\cdot\vecb{Z}^\dag\vecb{\s}\vecb{Z} \comma\\
						\DD_\mu &=\pd_\mu - \,\ii A_\mu -\ii\k\vecb{\s}_\mu\comma \\
						A_\mu &=-\frac{\ii}{2}\left[\vecb{Z}^\dag \left(\pd_\mu\!\vecb{Z}\right) - \left(\pd_\mu\!\vecb{Z}^\dag\right)\vecb{Z} \right] \comma
					\end{aligned}
				\right.
			\end{equation}
			其中的 $\k=D/2J$ 是DM耦合。
			
			在 $B=0$ 情况下,当初始 $\vecb{Z}_0$ 为任意常矢量时,对式~\eqref{EQ_GL_FUNCTIONAL_2} 变分,解得
			\begin{equation}
				\vecb{Z}_\text{H}=\ee^{\ii \k \left(\vecb{\s}\cdot\hat{\vecb{k}}\right) \left(\vecb{r}\cdot\hat{\vecb{k}}\right) } \vecb{Z}_0 \comma
			\end{equation}
			其中, $\vecb{r}$ 为平面上的坐标,而 $\vecb{k}$ 是与 $\vecb{n}_0=\vecb{Z}_0^\dag \vecb{\s}\vecb{Z}_0$ 相垂直的一个单位矢量。可验证 $\vecb{n}_\text{H}=\vecb{Z}_\text{H}^\dag \vecb{\s}\vecb{Z}_\text{H}$ 也与 $\vecb{k}$ 垂直,此时即是沿 $\vecb{k}$ 方向旋转的螺旋相。进一步计算可知,拓扑指数为0.
			
		\subsection{Skyrmion解}
			下面研究二维空间中一个单独的skyrmion。设其在中心 $r=0$ 处自旋方向向下;而 $r\approach\infty$ 时,自旋方向向上。可以写出式~\eqref{EQ_GL_FUNCTIONAL_2} 解的形式:
			\begin{equation}
				\vecb{Z}_\text{Sk}=\ee^{\ii \,\left(\th(r)/2\right) \left(\vecb{\s}\cdot\hat{\vecb{r}} \times \hat{\vecb{Z}}\right)} \vecb{Z}_0 \comma
			\end{equation}
			其中 $\vecb{Z}_0=(1,\,0)$,$\vecb{r}$ 为面内的极坐标矢量,$\th(r)$则是倾角 $\th$ 与极坐标 $r$ 的光滑函数。可取一个典型解分析
			\begin{equation}
				\vecb{Z}_\text{Sk}=\frac{1}{\sqrt{R^2+r^2}}
				\begin{pmatrix}
				\ii \,x+y \\ -R
				\end{pmatrix}
				\fullstop
			\end{equation}
			面内自旋用 $(\Re[z_1],\,\Im[z_1])$表示。该结构即是一个涡旋结构。这个解对应的 $\vecb{n}=\vecb{Z}^\dag \vecb{\s}\vecb{Z}$ 即为skyrmion。作出图像,就得到图~\ref{FIG_SKYRMIONS_WIKI} 中的 (b)。\cite{臧佳栋2012凝聚态物理学中的拓扑现象}
			
	\section{总结}
		在这篇文章中,我们对Heisenberg模型进行深入的研究,发现低温下铁磁相的形成对Monte Carlo步数的要求很高;低温下的螺旋相也是如此。然而skyrmion则是一种极稳定的相,很快就可以形成。我们认为skyrmion的形成是因为体系DM作用的存在而打破了 $xy$ 平面的空间反演对称,这使得破坏空间反演对称的skymion相可以大面积存在。这种拓扑结构,可以阻碍随温度增加时由于对称性增加而产生的体系熵的增加。换而言之,拓扑结构是为了维持体系自身能级简并的产物,以此来对抗温度变化而产生的熵的增加。这可以使体系维持在更低能量状态。这是该相如此稳定的原因。我们进一步希望研究其他相互作用对体系对称性的影响。
		
	\section{致谢}
		首先要感谢陈焱老师在全过程的帮助,尤其是对我们选题的指导。其次要感谢吴义政老师和虞跃老师,他们分别他从实验和理论的角度提出了一些很有价值的观点。还要感谢左光宏老师,他帮助我们实现了并行计算,并提供了服务器。最后要感谢张宇、王敬淳、秦金谷、詹晨迪和郑益峰等学长学姐的支持与帮助。
		
%	\section{参考文献}
		\nocite{谢锡麟2014现代张量分析,LippmanLajoieMooEtAl2013}
		{
			\flushleft \small
			\bibliography{Reference}
		}
	\end{multicols}
	\appendix
	\section{评分}
		杜治纬(20分)\quad 查阅Heisenberg模型的相关文献,撰写引言,运行程序,处理数据;
		
		孙晓晨(20分)\quad 查阅DM作用相关文献,负责引言以及DM相互作用的分析部分,程序debug,运行程序;
		
		陶润恺(20分)\quad C++、LabVIEW\textregistered 模拟Heisenberg模型,处理非线性$\sigma$模型部分,完善附录;
		
		王金银(20分)\quad 查阅skyrmion的相关文献,运行程序,处理数据,程序debug,撰写论文的总结与摘要;
		
		曾祥东(20分)\quad C++模拟skyrmion的产生,程序优化,结果推广,Mathematica\textregistered 图像处理,\LaTeX 论文排版。
		
	\section{Monte Carlo模拟的演化情况} \label{SEC_演化情况}
		对于经典Heisenberg模型的Metropolis模拟,点阵尺寸大约在 $14\times14$ 时,存在一个临界点。边长小于14时,能量与磁矩均迅速趋于稳定;大于14时,能量仍迅速稳定,而磁矩的收敛速度却大幅下降。在14附近,磁矩的演化存在突跃,见图~\ref{FIG_DYNAMIC}。其原因不明。
		\begin{myFigure}
			\includegraphics[width = 10cm]{Figures/FIG_DYNAMIC}
			\vspace{0.6cm}
			\figCaption[步数为 \num{40000}。$J=1$,$T=0$。]{平均能量、平均磁矩的演化曲线}
			\label{FIG_DYNAMIC}
		\end{myFigure}
		
		尺寸更大时,收敛速度进一步下降。若使用 $36\times36$ 的点阵,估计约需要300万MCS,磁矩才可稳定。在这种瞬态过程中,磁矩的具体数值有赖于初始情况,如图~\ref{FIG_DYNAMIC_36}。这就是导致图~\ref{FIG_HEISENBERG_E_M_VS_T_LATTICE_36} 中平均磁矩涨落巨大的原因。
		\begin{myFigure}
			\includegraphics[width = 5.5cm]{Figures/FIG_DYNAMIC_36}
			\vspace{0.6cm}
			\figCaption[步数为 \num{40000},点阵大小为 $36\times36$ ;$J=1$,$T=0$。三条曲线分别是三次计算的结果。相比图~\ref{FIG_DYNAMIC} 中的结果,这里的平均磁矩要小很多。]{平均磁矩的演化曲线}
			\label{FIG_DYNAMIC_36}
		\end{myFigure}
		
	\section{额外的图片} \label{SEC_额外的图片}
		\begin{myFigure}
			\vspace{-1cm}
			\includegraphics[width = \textwidth]{Figures/FIG_VECTOR_3D}
			\vspace{-0.8cm}
			\figCaption[从左至右依次为锥形相、skyrmion相和铁磁相。只选取了点阵中的部分自旋矢量作图。]{不同相的三维矢量图}
			\label{FIG_VECTOR_3D}
		\end{myFigure}
		\begin{myFigure}
			\vspace{-1.5cm}
			\includegraphics[width = \textwidth]{Figures/FIG_ISING_XY_VECTOR_3D}
			\vspace{-0.8cm}
			\figCaption[从左至右依次为极端Ising情形、Heisenberg情形和极端XY情形,各向异性指数 $\alpha$ 分别为100、1和0.01。其skyrmion密度分别为 \num{-9.72e-2}、\num{-7.07e-2} 和 \num{-5.49e-3}。各图中箭头的比例略有不同。]{不同模型的三维矢量图}
			\label{FIG_ISING_XY_VECTOR_3D}
		\end{myFigure}
		
	\section{拓扑不变量的证明} \label{SEC_拓扑不变量的证明}
		以下证明采用郭仲衡先生发明的张量分析语言表述。
		
		设 $\hat{\vecb{M}}=M\!\!\myB{i}e\!\myB{i}$,其中 $e^1=e^r$,$e^2=e^\th$,$e^3=e^\vf$。这里Christoffel符号的具体表示为
		\begin{align}
			\Gamma(212) &= -\Gamma(221)=\frac{1}{r}\comma\\
			\Gamma(313) &= -\Gamma(331)=\frac{1}{r}\comma\\
			\Gamma(323) &= -\Gamma(332)=\frac{1}{r}\frac{\cos\th}{\sin\th} \fullstop
		\end{align}
		对 $\mu$ 分量求协变导数
		\begin{equation}
			\nabla\!\myB{\mu} \hat{\vecb{M}} =\left[\pd\myB{\mu} M\!\!\myB{i}-\Gamma\!\myB{\mu ij} M\!\!\myB{j} \right] e^i \fullstop
		\end{equation}
		把式~\eqref{EQ_TOPOLOGICAL_INVARIANT} 积分内的部分作分量展开:
		\begin{align} \label{EQ_COMPONENTS}
			\hat{\vecb{M}}\cdot\left( \frac{\pd}{\pd \th} \hat{\vecb{M}} \times \frac{\pd}{\pd \vf} \hat{\vecb{M}}\right)
			&= M\!\!\myB{1}\left[\pd \myB{2}M\!\!\myB{2}+M\!\!\myB{1}\right]
			\left[\pd \myB{3}M\!\!\myB{3}+M\!\!\myB{1}+\cot \th M\!\!\myB{2}\right] \notag\\ 
			&\phantom{=}+ M\!\!\myB{2}[\pd\myB{2}M\!\!\myB{3}[\pd\myB{3}M\!\!\myB{1}-M
			\myB{3}]\notag\\
			&\phantom{=}+M\!\!\myB{3}[\pd\myB{2}M\!\!\myB{1}-M\!\!\myB{2}][\pd\myB{3}M\!\!\myB{2}-\cot \th M\!\!\myB{3}] \notag\\
			&\phantom{=}-M\!\!\myB{1}[\pd\myB{2}M\!\!\myB{3}][\pd\myB{3}M\!\!\myB{2}-\cot\th M\!\!\myB{3}] \notag\\
			&\phantom{=}-M\!\!\myB{2}[\pd\myB{2}M\!\!\myB{2}+M\!\!\myB{1}][\pd\myB{3}M\!\!\myB{3}+M\!\!\myB{1}+\cot\th M\!\!\myB{2} \notag\\
			&\phantom{=}-M\!\!\myB{3}[\pd\myB{2}M\!\!\myB{2}+M\!\!\myB{1}][\pd\myB{3}M\!\!\myB{1}-M\!\!\myB{3}] \fullstop
		\end{align}
		
		\begin{myFigure}
			\includegraphics[width = 4cm]{Figures/FIG_CIRCLE}
			\vspace{0.6cm}
			\figCaption[把skyrmion所在的 $E^2$ 底流形紧致成 $S^2$,其上可负载拓扑非平庸场。图中的圆(实际是球)代表 $S^2$ 流形。]{参考图示}
		\end{myFigure}
		
		第一种情况:设 $\hat{\vecb{M}}\perp e\!\myB{3}$,即$M\!\!\myB{3}=0$。$\abs{\hat{\vecb{M}}}=1$要求$M\!\!\myB{1}=\cos\omega\th$ 和$M\!\!\myB{2}=\sin\omega\th$;自洽要求$M\!\!\myB{2}\bigg|_{\th=\p}=\sin\omega\p=0$。所以$\omega\p=k\p,\,k=0,\,\pm1,\,\pm2\cdots$,故$\omega=k$。
		
		代入式~\eqref{EQ_COMPONENTS} 并计算,可得
		\begin{align}
			\hat{\vecb{M}}\cdot\left( \frac{\pd}{\pd \th} \hat{\vecb{M}} \times \frac{\pd}{\pd \vf} \hat{\vecb{M}}\right)
			&=\cos k\th \left(\pd_\th \sin k\th + \cos k\th\right) \left(\cos k\th + \cot\th \sin k\th\right) \notag\\
			&\phantom{=}-\sin k\th \left(\pd_\th \cos k\th - \sin k\th\right) \left(\cos k\th + \cot\th \sin k\th\right) \notag\\
			&=(1+k)(\cos k\th+\cot\th \sin k\th) \fullstop
		\end{align}
		拓扑指数 $Q$ 为该式在底流形上的积分:
		\begin{align}
			\frac{1}{8\p}\int \hat{\vecb{M}}\cdot\left( \frac{\pd}{\pd \th} \hat{\vecb{M}} \times \frac{\pd}{\pd \vf} \hat{\vecb{M}}\right)\sin\th\dd\th\dd\vf
			&=\frac{1}{2} \int_{0}^{\p} (1+k)[\cos k\th+\cot\th \sin k\th]\sin\th\dd\th \notag\\
			&=\frac{1+k}{2} \int_{0}^{\p} \left(\cos\th \sin k\th+ \sin\th \cos k\th\right) \dd\th \notag\\
			&=\frac{1+k}{2} \int_{0}^{\p} \sin (k+1)\th \dd\th \notag\\
			&=\frac{1}{2} \int_{0}^{(k+1)\p} \sin x \dd x \tag*{式中$x=(k+1)\th$。}\\
			&=\frac{1}{2} \left[1-\cos(k+1)\p\right] 
			=\frac{1+(-1)^k}{2}\comma
		\end{align}
		它是整数。
		
		第二种情况:设 $\hat{\vecb{M}}\perp e\myB{2}$,即$M\!\!\myB{2}=0$。同样,也有$\abs{\hat{\vecb{M}}}=1$要求$M\!\!\myB{1}=\cos\omega\th$ 和$M\!\!\myB{2}=\sin\omega\th$。所以$\omega=k$。同理可得
		\begin{align}
			\hat{\vecb{M}}\cdot\left( \frac{\pd}{\pd \th} \hat{\vecb{M}} \times \frac{\pd}{\pd \vf} \hat{\vecb{M}}\right)
			&=\cos k\th \cdot \cos k\th \cdot \cos k\th \notag\\
			&\phantom{=}+\sin k\th \left(\pd_\th \cos k\th\right) \left(-\cot\th \sin k\th\right) \notag\\
			&\phantom{=}-\cos k\th \left(\pd_\th \sin k\th\right) \left(-\cot\th \sin k\th\right) \notag\\
			&\phantom{=}-\sin k\th \cdot \cos k\th \cdot \left(-\sin k\th\right) \notag\\
			&=(1+k)(\cos k\th+\cot\th \sin k\th) \fullstop
		\end{align}
		积分得到拓扑指数:
		\begin{align}
			\frac{1}{8\p}\int \hat{\vecb{M}}\cdot\left( \frac{\pd}{\pd \th} \hat{\vecb{M}} \times \frac{\pd}{\pd \vf} \hat{\vecb{M}}\right)\sin\th\dd\th\dd\vf
			&=\frac{1}{2}\int_{0}^{\p} (\cos k\th+k\cot\th \sin k\th)\sin\th\dd\th \notag\\
			&=\frac{1+(-1)^k}{2}\comma
		\end{align}
		它也是整数。
		
		这两种情况分别对应图~\ref{FIG_SKYRMIONS_WIKI} 中的两种skyrmion。
		
	\section{部分代码实现}
		Heisenberg模型与产生skyrmion的系统,差别仅在于Hamilton量,因此其代码是相同的。为了加快运行速度,在模拟经典Heisenberg模型时,涉及其他相互作用的代码未被编译。
		
		为节省篇幅,这里只给出了核心代码,完整代码请参见 \url{https://github.com/Stone-Zeng/Thermo_Stat_Project}。编译平台:Microsoft\textregistered\;Visual C++ 2015。
		{
		\linespread{1}
%		\subsection{Head.h}
%		\lstinputlisting{../HeisenbergModel/Head.h}
		\subsection{Function.h}
		\lstinputlisting{../HeisenbergModel/Function.h}
		\subsection{MyLattice.h}
		\lstinputlisting{../HeisenbergModel/MyLattice.h}
		\subsection{MyLattice.cpp}
		\lstinputlisting{../HeisenbergModel/MyLattice.cpp}
%		\subsection{Physics.h}
%		\lstinputlisting{../HeisenbergModel/Physics.h}
%		\subsection{Physics.cpp}
%		\lstinputlisting{../HeisenbergModel/Physics.cpp}
		\subsection{SingleSimulation.h}
		\lstinputlisting{../HeisenbergModel/SingleSimulation.h}
		\subsection{SingleSimulation.cpp}
		\lstinputlisting{../HeisenbergModel/SingleSimulation.cpp}
%		\subsection{Main.cpp}
%		\lstinputlisting{../HeisenbergModel/HeisenbergModel_Main.cpp}
%		\subsection{PhaseDiagram.cpp}
%		\lstinputlisting{../HeisenbergModel/PhaseDiagram.cpp}
		}
		
	\section{LabVIEW\texorpdfstring{\textsuperscript{\textregistered}}{®}平台下的模拟结果}
		我们还利用LabVIEW\textregistered 进行了Heisenberg模型的模拟。正文使用的C++代码使用并行算法(模拟退火),而这里改用了串行算法,即在体系达到平衡后连续改变温度,并记录结果,如图~\ref{FIG_LABVIEW}。所使用的点阵大小为 $50\times50$,$J$ 取1。从初始温度 $T=0$ 开始,约1000MCS后,认为系统达到稳定。此后逐渐升温,每个温度下认为100MCS后稳定。其结果与正文中的类似。
		\begin{myFigure}
			\includegraphics[width = 12cm]{Figures/FIG_LABVIEW}
			\vspace{0.5cm}
			\figCaption[能量曲线中,灰色曲线为原始数据,蓝色曲线为平滑后的结果。热容曲线是利用能量曲线的差分来计算的。]{平均能量与热容随温度的变化曲线}
			\label{FIG_LABVIEW}
		\end{myFigure}
\end{document}