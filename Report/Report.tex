\documentclass{article}

\usepackage{geometry}
%	\geometry{a4paper, twoside, left = 2.54 cm, right = 2.54 cm, top = 2.54 cm, bottom = 2.54 cm, headheight = 3 cm}
	\geometry{a4paper, twoside, left = 1.91 cm, right = 1.91 cm, top = 2.54 cm, bottom = 2.54 cm, headheight = 3 cm}

\usepackage{titlesec}

\usepackage{xcolor}
\usepackage[hyperfootnotes=false]{hyperref}
	\hypersetup{
		bookmarksopen = true,
		bookmarksopenlevel = 1,
		bookmarksnumbered = true,
		pdftitle = {基于Heisenberg模型的Skyrmion研究},%TODO:title
		pdfauthor = {杜治纬,孙晓晨,陶润恺,王金银,曾祥东},
		colorlinks,
		linkcolor = {red!60!black},
		citecolor = {green!50!black},
		urlcolor = {blue!70!black}
	}

\usepackage[stable, perpage]{footmisc}
\usepackage{pifont}
	\renewcommand{\thefootnote}{\ding{\numexpr171+\value{footnote}}}

\usepackage[numbers, square, super]{natbib}
	\setlength{\bibsep}{0 pt}
	\renewcommand{\citenumfont}{\itshape}
	\renewcommand{\bibsection}{\section{参考文献}}
	\bibliographystyle{gbt-7714-2015-numerical}

\usepackage{amsmath}
\usepackage{mathtools}
\usepackage[adobe-garamond]{mathdesign}
\usepackage{bm}

\usepackage[no-math]{fontspec}
	\setmainfont[Ligatures = TeX, BoldFont = AGaramondPro-Semibold.otf]{AGaramondPro-Regular.otf}

\usepackage[UTF8, heading = true, 10pt]{ctex}
\usepackage{xeCJK}
	\xeCJKsetup{
		AutoFakeBold = false,
		AutoFakeSlant = false
	}
	\setCJKmainfont[BoldFont = 华文中宋, ItalicFont = 华文楷体, Mapping = fullwidth-stop]{宋体}
	\setCJKfamilyfont{中宋}[Mapping = fullwidth-stop]{华文中宋}
	\setCJKfamilyfont{楷体}[Mapping = fullwidth-stop]{华文楷体}
	\setCJKfamilyfont{仿宋}[Mapping = fullwidth-stop]{华文仿宋}
	\setCJKfamilyfont{黑体}[BoldFont = 黑体, Mapping = fullwidth-stop]{黑体}
	
	\newfontfamily \courier {Courier New}
	\newfontfamily \verdana {Verdana}
	\newfontfamily \helvetica {HelveticaNeueLTPro-Roman.otf}[BoldFont = HelveticaNeueLTPro-Roman.otf]
	
	\newcommand{\myHeavy}{\helvetica \CJKfamily{黑体}}
	\renewcommand{\sffamily}{\verdana}
	\renewcommand{\ttfamily}{\courier}
%	\setcounter{secnumdepth}{4}
	\pagestyle{plain}
	\ctexset{
		%abstractname = {中},%{\myHeavy \normalsize 摘\phantom{空}要},
		appendix = {
			name = {附录}
		},
		section = {
			format = {\myHeavy \Large \centering},
			name = {,、\hspace{-1 em}},
			number = \chinese{section},
		},
		subsection = {
			format = {\myHeavy \large},
%			name = {,、\hspace{-1 em}},
%			numbering = true,
%			number = \chinese{subsection},
		}
	}
	\renewcommand{\abstractname}{\myHeavy \large 摘 \quad 要}

\usepackage{enumitem}

\usepackage{multicol}
\setlength{\columnsep}{20 pt}

\usepackage{listings}
\lstset{
	language = C++,
	basicstyle = \ttfamily \footnotesize,
	breaklines = true,
	tabsize = 4,
	numbers = left,
	numberstyle = \sffamily \itshape \tiny \color{gray},
	numbersep = 10 pt,
	showstringspaces = false,
	keywordstyle = \bfseries \color[rgb]{0,0,1},
	commentstyle = \color[rgb]{0,0.5,0},
	stringstyle = \color[rgb]{0.64,0.082,0.082},
	morecomment = [l][\color{violet}]{\#}
}

\usepackage{array}
	\newcolumntype{M}{>{$}c<{$}} %数学模式,居中
\usepackage{tabularx}
	\newcolumntype{Y}{>{\centering\arraybackslash}X} %定宽居中
\usepackage{booktabs} %三线表

\usepackage{graphicx}

\usepackage{float} %浮动体H选项
\usepackage{caption}
	\DeclareCaptionFont{myFigureCaptionFont}{\small}
	\DeclareCaptionFont{myTableCaptionFont}{\myHeavy \small}
	\captionsetup[figure]{
		font = myFigureCaptionFont,
		labelsep = quad,
		skip = 10 pt,
		position = bottom,
%		justification = centerfirst
		format = hang
	}
	\captionsetup[table]{
		font = myTableCaptionFont,
		labelsep = quad,
		skip = 10 pt,
		position = top
	}
\usepackage[labelformat=simple]{subcaption} %并排图表
\makeatletter
\renewcommand\p@subfigure{\thefigure-}
\renewcommand\thesubfigure{(\alph{subfigure})} %Subfig reference format: x-(x)
\makeatother

%大写正体字母
\newcommand{\rmY}{\mathrm{Y}}

%大写手写体字母
\newcommand{\scF}{\mathcal{F}}
\newcommand{\scH}{\mathcal{H}}
\newcommand{\scI}{\mathcal{I}}
\newcommand{\scJ}{\mathcal{J}}
\newcommand{\scL}{\mathcal{L}}
\newcommand{\scP}{\mathcal{P}}
\newcommand{\scR}{\mathcal{R}}
\newcommand{\scS}{\mathcal{S}}

%小写希腊字母
\renewcommand{\a}{\alpha}
\renewcommand{\b}{\beta}
\newcommand{\g}{\gamma}
\renewcommand{\d}{\delta}
\newcommand{\e}{\epsilon}
\newcommand{\ve}{\varepsilon}
\newcommand{\z}{\zeta}
\newcommand{\h}{\eta}
\newcommand{\q}{\theta}
\renewcommand{\th}{\theta}
\newcommand{\vq}{\vartheta}
\newcommand{\vth}{\vartheta}
\renewcommand{\i}{\iota}
\renewcommand{\k}{\varkappa} %这两个是相反的
\newcommand{\vk}{\kappa} %这两个是相反的
\renewcommand{\l}{\lambda}
\renewcommand{\m}{\mu}
\newcommand{\n}{\nu}
\newcommand{\x}{\xi}
%\renewcommand{\p}{\pi} %\p 已定义为正体π
\newcommand{\vp}{\varpi}
\renewcommand{\r}{\rho}
\newcommand{\vr}{\varrho}
\newcommand{\s}{\sigma}
\newcommand{\vs}{\varsigma}
\renewcommand{\t}{\tau}
\renewcommand{\u}{\upsilon}
\newcommand{\f}{\phi}
\renewcommand{\j}{\varphi}
\newcommand{\vf}{\varphi}
\newcommand{\irchi}[2]{\raisebox{\depth}{$#1\chi$}}
\renewcommand{\c}{{\mathpalette\irchi\relax}}
\newcommand{\y}{\psi}
\renewcommand{\o}{\omega}
\newcommand{\w}{\omega}

%大写希腊字母
\renewcommand{\G}{\varGamma}
\newcommand{\D}{\varDelta}
\newcommand{\Q}{\varTheta}
\newcommand{\Th}{\varTheta}
\renewcommand{\L}{\varLambda}
\newcommand{\X}{\varXi}
\renewcommand{\P}{\varPi}
\let \secSymbol = \S
\renewcommand{\S}{\varSigma}
\renewcommand{\U}{\varUpsilon}
\newcommand{\F}{\varPhi}
\newcommand{\Y}{\varPsi}
\renewcommand{\O}{\varOmega}
\newcommand{\W}{\varOmega}

%常数
\newcommand{\p}{\piup} %圆周率
\newcommand{\ee}{\mathrm{e}} %自然对数基
\newcommand{\ii}{\mathrm{i}} %虚数单位
\newcommand{\hb}{\hbar} %约化Planck常数
\newcommand{\kB}{k_{\mathrm{B}}} %Boltzmann常数
\newcommand{\pfs}{\e_0} %真空介电常数 Permittivity of Free Space

%数学修饰符号
\newcommand{\vecb}[1]{\bm{#1}} %矢量用斜体,根据ISO 80000-2/2009
\renewcommand{\bar}[1]{\overline{#1}} %上横线
\newcommand{\tl}[1]{\widetilde{#1}} %弯号

%数学函数
\newcommand{\abs}[1]{\left\lvert #1 \right\rvert} %绝对值
\newcommand{\mean}[1]{\left< #1 \right>} %统计平均

%算子、算符
%已经调整好前后间距
\DeclareMathOperator{\incr}{\Delta\!}
\DeclareMathOperator{\dd}{\mathrm{d}\!} %微分 Differential D
\DeclareMathOperator{\db}{{\mkern2.5mu\mathchar'26\mkern-11mu\mathrm{d}}\!} %不完全微分 Inexact Differential D; Only for "Adobe Garamond" font!
\DeclareMathOperator{\pd}{\partial\!} %偏微分 Partial D
\DeclareMathOperator{\vd}{\deltaup\!} %变分 Variational D
\DeclareMathOperator{\laplac}{\nabla^2\!} %Laplace算子

%复合数学符号
\newcommand{\infint}{\int_{-\infty}^{+\infty}} %上下限无穷的积分
\newcommand{\iTonSum}{\sum_{i = 1}^{n}} %求和 i=1..n
\newcommand{\iToNSum}{\sum_{i = 1}^{N}} %求和 i=1..N
\newcommand{\nEnum}[2]{#1 = 1, \, 2, \, \dots, \, #2} %如x=1,2,...,n
%\newcommand{\coordinateLabel}[2]{#1\text{-}#2} %如p-V坐标
\newcommand{\myPartial}[3]{\left( \frac{\pd #1}{\pd #2} \right)_{#3}} %带括号的偏微分
\newcommand{\myPartialDisplay}[3]{\left( \dfrac{\pd #1}{\pd #2} \right)_{#3}}

%重定义符号
\newcommand{\approach}{\rightarrow} %趋向于
\newcommand{\eqdef}{\equiv} %定义等号(三线)
\newcommand{\dotTimes}{\cdot} %点乘

%公式用标点、文字
\newcommand{\comma}{\text{,}}
\newcommand{\fullstop}{\text{.}}
\newcommand{\semicomma}{\text{;}}
\newcommand{\const}{\text{const.}}

%正文用符号
%\catcode`\。 = \active
%\newcommand{。}{.} %句号用句点代替,已用xeCJK的Mapping代替
%\newcommand{\mySec}{{\fontspec{Times New Roman} \S \,}} %TODO:20160302 冲突;章节符号,Garamond 字体不好看

\newcommand{\emphA}[1]{{\myHeavy #1}}
\newcommand{\emphB}[1]{{\itshape #1}}
\newcommand{\figCaption}[2][]{\vspace{-0.8cm}\captionof{figure}{#2 \\ \emphB{#1}}}

\newenvironment{myAbstract}
	{\begin{abstract} \normalsize}
	{\end{abstract}}

\newenvironment{myFigure}
	{\par\medskip\noindent\minipage{\linewidth}\centering}
	{\endminipage\par\medskip}

\title{
	\vspace{-2 cm} \LARGE \bfseries
	基于Heisenberg模型的Skyrmion研究
}
\author{
	\CJKfamily{楷体}
	杜治纬,孙晓晨,陶润恺,王金银,曾祥东
}

\date{
	\CJKfamily{楷体}
	\today
}

\begin{document}
	\maketitle
	
	\begin{myAbstract}
		我们
	\end{myAbstract}
	
	\begin{multicols}{2}
	\section{引言}
		Skyrmion最初由Skyrme在1962年提出 ,用于刻画描述重子的行为。它是一种具有特定拓扑结构的准粒子,并被实验观察所证实\cite{yu2010real}。
		
		在磁性物质中,如果各电子自旋方向满足Skyrmion解,那就会形成一种稳定且具有手征性和拓扑性的磁结构。在二维磁性系统中,可以形成的一种Skyrmion呈涡旋形,自旋方向由中心向外周依次偏转,中心位置自旋向下,到最外周处自旋统一向上,如图 \ref{FIG_Skyrmions_WIKI} 所示。它的尺度通常是纳米数量级\cite{ulrich2011chiral}。由于自身的拓扑性质,Skyrmion能在外界扰动下保持稳定。%TODO:图,到底是哪一种
		
		\begin{myFigure}
			\includegraphics[width = 5 cm]{"Figures/FIG_Skyrmions_WIKI.png"}
			\vspace{0.8cm}
			\figCaption[图 (a) 反映的是“刺猬”型Skyrmion,图 (b) 反映的是涡旋型Skyrmion。]{二维Skyrmion的示意图\cite{wiki:MagneticSkyrmion}}
			\label{FIG_Skyrmions_WIKI}
		\end{myFigure}
		
		微观尺度上,Skyrmion的形成源于电子间的Dzyaloshinsky--Moriya(DM)相互作用\cite{dzyaloshinsky1958thermodynamic,moriya1960anisotropic,wiki:MagneticSkyrmion}。它是由于界面附近空间对称性破坏而产生的,对应的Hamilton量可以表示为
		\begin{equation}
			\scH_\text{DM} = \vecb{D}_{12} \cdot (\vecb{s}_1 \times \vecb{s}_2) \comma
		\end{equation}
		其中的 $\vecb{s}_1$、$\vecb{s}_2$ 分别表示两个电子的自旋矢量,而 $\vecb{D}_{12}$ 是一个常矢量,由界面材料和晶体结构决定。
	\section{Heisenberg模型}
		\subsection{介绍}
			(经典)Heisenberg模型是开展Skyrmion等一系列研究的基础,它是 $n$-矢量模型在 $n = 3$ 时的情况。取一个 $d$ 维点阵,每个点阵点均是具有单位长度的三维自旋矢量
			\begin{equation}
				\vecb{s}_i \in \mathbb{R}^3, \quad \abs{\vecb{s}_i} = 1 \fullstop
			\end{equation}
			若系统的Hamilton量定义为
			\begin{equation} \label{EQ_DEF_OF_HEISENBERG_MODEL}
				\scH = -\sum_{\mean{i,j}} J_{i\!j} \vecb{s}_i \cdot \vecb{s}_j \comma
			\end{equation}
			则该自旋系统就是Heisenberg模型\cite{joyce1967classical,wiki:ClassicalHeisenbergModel}。式~\eqref{EQ_DEF_OF_HEISENBERG_MODEL} 中的
			\begin{equation}
				J_{i\!j} =
				\begin{cases}
					J, & \text{若 $i$、$j$ 为近邻} \semicomma \\
					0, & \text{其他情况} \comma
				\end{cases}
			\end{equation}
			在Hamilton量中添加其他的相互作用,就可以得到各种外场下的情况。
			
		\subsection{算法}
			我们研究的是二维Heisenberg模型,并且不考虑外场。利用Metropolis算法(一种Monte Carlo算法),可以对该模型进行模拟。该算法的思路如下:
			\begin{enumerate}[itemsep = 0 pt]
				\item 给出自旋矢量在平面点阵中的一个随机分布。
				\item 选择一个自旋 $\vecb{s}_i$。
				\item 随机生成一个自旋矢量 $\vecb{s}'_i$,计算从 $\vecb{s}_i$ 翻转到 $\vecb{s}'_i$ 所带来的的能量变化 $\incr E$,从而得到翻转概率
					\begin{equation}
						P =
						\begin{cases}
							\exp \left(-\dfrac{\incr E}{\kB T}\right), &\incr E > 0 \semicomma \\
							1, &\incr E \leqslant 0 \comma
						\end{cases}
					\end{equation}
				\item 生成一个 $[0,1)$ 之间的随机数。若该随机数小于翻转概率 $P$,则将 $\vecb{s}_i$ 翻转为 $\vecb{s}'_i$。
				\item 重复进行步骤 $2\sim4$。我们采取的方式是遍历整个点阵,将它记为一个Monte Carlo步(MCS)。进行若干MCS后,体系就可以达到稳态。
				\item 在最后的若干MCS中计算各物理量,并取其平均值。\cite{周琼2010蒙特卡洛方法在磁性系统中的应用}%TODO:为什么
			\end{enumerate}
			
			在我们的模拟中,点阵大小取为 $36 \times 36$。整个过程包含1000个MCS,并在最后50步中计算各物理量。另外,所有参数均采用约化单位。其中,取 $\kB=1$,$J=1$。
			
			下面给出各物理量的计算公式。总能量
			\begin{equation}%TODO:系数
				E = \sum_{i} \ve_i \comma
			\end{equation}
			这里的 $\ve_i$ 即单个格点的Hamilton量。总磁矩
			\begin{equation}
				\vecb{M} = \sum_{i} \vecb{s}_i \fullstop
			\end{equation}
			比热
			\begin{equation} \label{EQ_HEAT_CAPACITY}
				C = \frac{\mean{E^2} - \mean{E}^2}{\kB T^2}
				= \frac{1}{\kB T^2} \left[ \frac{\sum_{i} \ve_i^2}{L^2} - \left( \frac{\sum_{i} \ve_i}{L^2} \right)^2 \right] \semicomma
			\end{equation}
			磁化率
			\begin{equation} \label{EQ_MAGNETIC_SUSCEPTIBILITY}
				\c = \frac{\mean{\vecb{M}^2} - \mean{\vecb{M}}^2}{\kB T}
				= \frac{1}{\kB T} \left[ \frac{\sum_{i} \vecb{s}_i^2}{L^2} - \left( \frac{\sum_{i} \vecb{s}_i}{L^2} \right)^2 \right] \fullstop
			\end{equation}
			以上两式中的 $L$ 为点阵边长。实际模拟中,各常数均忽略不计。
			
		\subsection{结果与分析}
			温度 $T$ 从0取到4,测量能量 $E$、总磁矩 $\vecb{M}$、比热 $C$ 和磁化率 $\c$ 随温度的变化。
			\begin{myFigure}
				\includegraphics[width = \textwidth]{Figures/FIG_HEISENBERG_E_M_VS_T}
				\figCaption[灰色曲线表示原始数据,蓝色曲线表示平滑后的结果;数据已进行单位化,即将原始值除以格点数。]{能量、总磁矩大小随温度的变化}
			\end{myFigure}
			
			由于式 \eqref{EQ_HEAT_CAPACITY} 和 \eqref{EQ_MAGNETIC_SUSCEPTIBILITY} 在 $T \approach 0$ 时是发散的,因此热容与磁化率利用能量、总磁矩曲线的导数来计算,如图 
			\ref{FIG_HEISENBERG_C_X_VS_T} 中的蓝色曲线所示。
			\begin{myFigure}
				\includegraphics[width = \textwidth]{Figures/FIG_HEISENBERG_C_X_VS_T}
				\figCaption[灰色曲线表示原始数据,蓝色曲线表示利用能量、总磁矩曲线的导数(差分)计算出的结果。]{热容、磁化率随温度的变化}
				\label{FIG_HEISENBERG_C_X_VS_T}
			\end{myFigure}
			
			要写分析!!!
	\section{Skyrmion的模拟}
		\subsection{介绍}
			连续情况下,Hamilton量为\cite{nagaosa2013topological}
			\begin{equation}
				\scH = \int \dd r \left[ J(\nabla \vecb{n})^2 + D \vecb{n} \cdot (\nabla \times \vecb{n}) - \vecb{H} \cdot \vecb{n} \right] \comma
			\end{equation}
			式中的 $J$ 是交换作用常数($J>0$ 时为铁磁相互作用),$D$ 是DM相互作用常数,$\vecb{H}$ 是外加磁场。
			
			将其离散化,可得\cite{wiki:MagneticSkyrmion}
			\begin{equation}
				\begin{aligned}
					\scH=&-J\sum_{\vecb{r}}\vecb{s}_{\vecb{r}} \cdot
					\left(\vecb{s}_{\vecb{r}\pm\vecb{e}_x}\!+\vecb{s}_{\vecb{r}\pm\vecb{e}_y} \right) \\
					&-D\sum_{\vecb{r}} \left[\vecb{s}_{\vecb{r}}\times\vecb{s}_{\vecb{r}\pm\vecb{e}_x}\!\!\cdot\left(\pm\vecb{e}_x\right) + \vecb{s}_{\vecb{r}}\times\vecb{s}_{\vecb{r}\pm\vecb{e}_y}\!\!\cdot\left(\pm\vecb{e}_y\right) \right] \\
					&-\vecb{H}\cdot\sum_{\vecb{r}}\vecb{s}_{\vecb{r}} \comma
				\end{aligned}
			\end{equation}
			其中,第一项为交换作用,即式~\eqref{EQ_DEF_OF_HEISENBERG_MODEL};第二、第三项分别为DM相互作用和外加磁场。
			
			为了定量描述Skyrmion相的特性,除了能量与磁矩,还需引进Skyrmion密度(SkD)\cite{muhlbauer2009skyrmion}
			\begin{equation}
				\f=\frac{1}{4\p}\vecb{s}\cdot \left(\frac{\pd \vecb{s}}{\pd x}\times\frac{\pd \vecb{s}}{\pd y}\right) \comma
			\end{equation}
			其中的 $x$、$y$ 是垂直于磁场 $\vecb{H}$ 的坐标分量。当 $\abs{\f}$ 接近1时,即表明出现了稳定的拓扑结构。将其离散化,并忽略系数,可得
			\begin{equation}
				\f=\sum_{\vecb{r}} \vecb{s}\cdot\left[ \left(\vecb{s}_{\vecb{r}+\vecb{e}_x}\!-\vecb{s}_{\vecb{r}-\vecb{e}_x}\right) \times\left(\vecb{s}_{\vecb{r}+\vecb{e}_y}\!-\vecb{s}_{\vecb{r}-\vecb{e}_y}\right)\right] \fullstop
			\end{equation}
			
			利用Monte Carlo算法模拟Skyrmion的思路与方法,和前文所述的Heisenberg模型完全相同。点阵大小同样也取为 $36\times36$。不同的是,由于存在较复杂的相互作用,达到稳定所需的步数会更多。
			
		\subsection{结果与分析}
			\begin{myFigure}
				\includegraphics[width = \textwidth]{Figures/FIG_PHASE_SD_AND_MZ}
				\figCaption{Skyrmion密度与磁矩 $z$ 方向分量随外加磁场 $\vecb{H}$ 和温度 $T$ 的变化}
			\end{myFigure}
			\begin{myFigure}
				\includegraphics[width = 7cm]{Figures/FIG_PHASE}
				\vspace{0.5cm}
				\figCaption{相图}
			\end{myFigure}
			\begin{myFigure}
				\includegraphics[width = \textwidth]{Figures/FIG_LATTICE_VECTOR}
%				\vspace{0.5cm}
				\figCaption{$xy$ 平面投影}
			\end{myFigure}
%			\begin{myFigure}
%				\includegraphics[width = 5cm]{Figures/FIG_VECTOR_SKX}
%				\vspace{0.5cm}
%				\figCaption{Skyrmion相}
%			\end{myFigure}
%			\begin{myFigure}
%				\includegraphics[width = 5cm]{Figures/FIG_VECTOR_FM}
%				\vspace{0.5cm}
%				\figCaption{铁磁相}
%			\end{myFigure}
			\begin{myFigure}
				\includegraphics[width = \textwidth]{Figures/FIG_LATTICE_MATRICX}
%				\vspace{0.5cm}
				\figCaption{$z$ 分量}
			\end{myFigure}
	\section{简要的数学}
	\section{总结}
	\section{致谢}
		首先感谢陈焱老师在全过程的帮助,尤其是对我们选题的指导。其次要感谢吴义政老师,他从实验的角度提出了一些很有价值的观点。还要感谢左光宏老师,他帮助我们实现了并行计算,并提供了服务器。最后要感谢张宇、秦金谷、詹晨迪和郑益峰等学长学姐的支持与帮助。
		
%	\section{参考文献}
		\nocite{LippmanLajoieMooEtAl2013}
		{
			\flushleft \small
			\bibliography{Reference}
		}
	\end{multicols}
	\appendix
	\section{评分}
	\section{代码实现}
		Heisenberg模型与产生Skyrmion的系统,差别仅在于Hamilton量,因此其代码是相同的。为了加快运行速度,在模拟经典Heisenberg模型时,涉及其他相互作用的代码未被编译。
		
		矢量类(\texttt{MyVector})、计时类(\texttt{MyTiming})和若干简单函数的实现代码没有给出。编译平台:Microsoft\textregistered\;Visual C++ 2015。
		\linespread{1}
		\subsection{Head.h}
		\lstinputlisting{../HeisenbergModel/Head.h}
		\subsection{MyLattice.h}
		\lstinputlisting{../HeisenbergModel/MyLattice.h}
		\subsection{MyLattice.cpp}
		\lstinputlisting{../HeisenbergModel/MyLattice.cpp}
		\subsection{Physics.h}
		\lstinputlisting{../HeisenbergModel/Physics.h}
		\subsection{Physics.cpp}
		\lstinputlisting{../HeisenbergModel/Physics.cpp}
		\subsection{SingleSimulation.h}
		\lstinputlisting{../HeisenbergModel/SingleSimulation.h}
		\subsection{SingleSimulation.cpp}
		\lstinputlisting{../HeisenbergModel/SingleSimulation.cpp}
		\subsection{Main.cpp}
		\lstinputlisting{../HeisenbergModel/HeisenbergModel_Main.cpp}
		\subsection{PhaseDiagram.cpp}
		\lstinputlisting{../HeisenbergModel/PhaseDiagram.cpp}
\end{document}