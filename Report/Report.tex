\documentclass{article}

\usepackage{geometry}
	\geometry{a4paper, twoside, left = 2.54 cm, right = 2.54 cm, top = 3.18 cm, bottom = 3.18 cm, headheight = 3 cm}

\usepackage{titlesec}

\usepackage{xcolor}
\usepackage[hyperfootnotes=false]{hyperref}
	\hypersetup{
		bookmarksopen = true,
		bookmarksopenlevel = 1,
		bookmarksnumbered = true,
		pdftitle = {基于Heisenberg模型的Skyrmion研究},%TODO:title
		pdfauthor = {杜治纬,孙晓晨,陶润恺,王金银,曾祥东},
		colorlinks,
		linkcolor = {red!60!black},
		citecolor = {green!50!black},
		urlcolor = {blue!70!black}
	}

\usepackage[stable, perpage]{footmisc}
\usepackage{pifont}
	\renewcommand{\thefootnote}{\ding{\numexpr171+\value{footnote}}}

\usepackage[numbers, square, super]{natbib}
	\setlength{\bibsep}{0 pt}
	\renewcommand{\citenumfont}{\itshape}
	\renewcommand{\bibsection}{\section{参考文献}}
	\bibliographystyle{gbt-7714-2015-numerical}

\usepackage{amsmath}
\usepackage{mathtools}
\usepackage[adobe-garamond]{mathdesign}
\usepackage{bm}

\usepackage[no-math]{fontspec}
	\setmainfont[Ligatures = TeX, BoldFont = AGaramondPro-Semibold.otf]{AGaramondPro-Regular.otf}

\usepackage[UTF8, heading = true]{ctex}
\usepackage{xeCJK}
	\xeCJKsetup{
		AutoFakeBold = false,
		AutoFakeSlant = false
	}
	\setCJKmainfont[BoldFont = 华文中宋, ItalicFont = 华文楷体, Mapping = fullwidth-stop]{宋体}
	\setCJKfamilyfont{中宋}[Mapping = fullwidth-stop]{华文中宋}
	\setCJKfamilyfont{楷体}[Mapping = fullwidth-stop]{华文楷体}
	\setCJKfamilyfont{仿宋}[Mapping = fullwidth-stop]{华文仿宋}
	\setCJKfamilyfont{黑体}[BoldFont = 黑体, Mapping = fullwidth-stop]{黑体}
	
	\newfontfamily \courier {Courier New}
	\newfontfamily \verdana {Verdana}
	\newfontfamily \helvetica {HelveticaNeueLTPro-Roman.otf}[BoldFont = HelveticaNeueLTPro-Roman.otf]
	
	\newcommand{\myHeavy}{\helvetica \CJKfamily{黑体}}
	\renewcommand{\sffamily}{\verdana}
	\renewcommand{\ttfamily}{\courier}
%	\setcounter{secnumdepth}{4}
	\pagestyle{plain}
	\ctexset{
		%abstractname = {中},%{\myHeavy \normalsize 摘\phantom{空}要},
		appendix = {
			name = {附录}
		},
		section = {
			format = {\myHeavy \Large \centering},
			name = {,、\hspace{-1 em}},
			number = \chinese{section},
		},
		subsection = {
			format = {\myHeavy \large},
%			name = {,、\hspace{-1 em}},
%			numbering = true,
%			number = \chinese{subsection},
		}
	}
	\renewcommand{\abstractname}{\myHeavy \large 摘 \quad 要}

\usepackage{enumitem}

\usepackage{listings}
\lstset{
	language = C++,
	basicstyle = \ttfamily \footnotesize,
	breaklines = true,
	tabsize = 4,
	numbers = left,
	numberstyle = \sffamily \tiny \color{gray},
	numbersep = 10 pt,
	showstringspaces = false,
	keywordstyle = \color[rgb]{0,0,1},
	commentstyle = \color[rgb]{0,0.5,0},
	stringstyle = \color[rgb]{0.64,0.082,0.082},
	morecomment = [l][\color{violet}]{\#}
}

\usepackage{array}
	\newcolumntype{M}{>{$}c<{$}} %数学模式,居中
\usepackage{tabularx}
	\newcolumntype{Y}{>{\centering\arraybackslash}X} %定宽居中
\usepackage{booktabs} %三线表

\usepackage{graphicx}

\usepackage{float} %浮动体H选项
\usepackage{caption}
	\DeclareCaptionFont{myFigureCaptionFont}{\small}
	\DeclareCaptionFont{myTableCaptionFont}{\myHeavy \small}
	\captionsetup[figure]{
		font = myFigureCaptionFont,
		labelsep = quad,
		skip = 10 pt,
		position = bottom
	}
	\captionsetup[table]{
		font = myTableCaptionFont,
		labelsep = quad,
		skip = 10 pt,
		position = top
	}
\usepackage[labelformat=simple]{subcaption} %并排图表

\input{Symbols.tex}

\newcommand{\emphA}[1]{{\myHeavy #1}}
\newcommand{\emphB}[1]{{\itshape #1}}

\newenvironment{myAbstract}
	{\begin{abstract} \normalsize}
	{\end{abstract}}

\title{
	\vspace{-2 cm} \LARGE \bfseries
	基于Heisenberg模型的Skyrmion研究
}
\author{
	\CJKfamily{楷体}
	杜治纬,孙晓晨,陶润恺,王金银,曾祥东
}

\date{
	\CJKfamily{楷体}
	\today
}

\begin{document}
	\maketitle
	
	\begin{myAbstract}
		四个米子
	\end{myAbstract}
	
	\section{引言}
	
	\section{Heisenberg模型}
		\subsection{介绍}
			(经典)Heisenberg模型是开展Skyrmion等一系列研究的基础,它是 $n$-矢量模型在 $n = 3$ 时的情况。取一个 $d$ 维点阵,每个点阵点均是具有单位长度的三维自旋矢量
			\begin{equation}
				\vecb{s}_i \in \mathbb{R}^3, \quad \abs{\vecb{s}_i} = 1 \fullstop
			\end{equation}
			若系统的Hamilton量定义为
			\begin{equation} \label{EQ_DEF_OF_HEISENBERG_MODEL}
				\scH = -\sum_{\mean{i,j}} J_{i\!j} \vecb{s}_i \cdot \vecb{s}_j \comma
			\end{equation}
			则该自旋系统就是Heisenberg模型。\cite{joyce1967classical,wiki:ClassicalHeisenbergModel} 式~\eqref{EQ_DEF_OF_HEISENBERG_MODEL} 中的
			\begin{equation}
				J_{i\!j} =
				\begin{cases}
					J, & \text{若 $i$、$j$ 为近邻} \semicomma \\
					0, & \text{其他情况} \comma
				\end{cases}
			\end{equation}
			在Hamilton量中添加其他的相互作用,就可以得到各种外场下的情况。
			
		\subsection{算法}
			我们研究的是二维Heisenberg模型,并且不考虑外场。利用Metropolis算法(一种Monte Carlo算法),可以对该模型进行模拟。该算法的思路如下:
			\begin{enumerate}
				\item 给出自旋矢量在平面点阵中的一个随机分布。
				\item 选择一个自旋 $\vecb{s}_i$。
				\item 随机生成一个自旋矢量 $\vecb{s}'_i$,计算从 $\vecb{s}_i$ 翻转到 $\vecb{s}'_i$ 所带来的的能量变化 $\incr E$,从而得到翻转概率
					\begin{equation}
						P =
						\begin{cases}
							\exp \left(-\frac{\incr E}{\kB T}\right), &\incr E > 0 \semicomma \\
							1, &\incr E \leqslant 0 \comma
						\end{cases}
					\end{equation}
				\item 生成一个 $[0,1)$ 之间的随机数。若该随机数小于翻转概率 $P$,则将 $\vecb{s}_i$ 翻转为 $\vecb{s}'_i$。
				\item 重复进行步骤 $2\sim4$。我们采取的方式是遍历整个点阵,将它记为一个Monte Carlo步(MCS)。进行若干MCS后,体系就可以达到稳态。
				\item 在最后的若干MCS中计算各物理量,并取其平均值。\cite{周琼2010蒙特卡洛方法在磁性系统中的应用}%TODO:为什么
			\end{enumerate}
			
			在我们的模拟中,点阵大小取为 $36 \times 36$。整个过程包含1000个MCS,并在最后50步中计算各物理量。另外,所有参数均采用约化单位。其中,取 $\kB=1$,$J=1$。
			
		\subsection{结果与分析}
			温度 $T$ 从0取到4,测量能量 $E$、总磁矩 $M$、比热 $C$ 和磁化率 $\c$ 随温度的变化。
	\section{Monte Carlo模拟}
	\section{简要的数学}
	\section{总结}
	\section{致谢}
		首先感谢陈焱老师在全过程的帮助,尤其是对我们选题的指导。其次要感谢吴义政老师,他从实验的角度提出了一些很有价值的观点。还要感谢左光宏老师,他帮助我们实现了并行计算。最后要感谢张宇学长、秦金谷学姐和李超然学长等人的支持与帮助,他们的科研经历给我们的课题带来了很多启发。
		
%	\section{参考文献}
		\nocite{LippmanLajoieMooEtAl2013}
		\flushleft \small
		\bibliography{Reference}
	
	\appendix
	\section{评分}
	\section{代码实现}
		\linespread{1}
%		\lstinputlisting{Codes/Head.h}
		\lstinputlisting{Codes/HeisenbergModel_Main.cpp}
\end{document}