\documentclass{article}

\usepackage{geometry}
%	\geometry{a4paper, twoside, left = 2.54 cm, right = 2.54 cm, top = 2.54 cm, bottom = 2.54 cm, headheight = 3 cm}
	\geometry{a4paper, twoside, left = 1.91 cm, right = 1.91 cm, top = 2.54 cm, bottom = 2.54 cm, headheight = 3 cm}

\usepackage{titlesec}

\usepackage{xcolor}
\usepackage[hyperfootnotes=false]{hyperref}
	\hypersetup{
		bookmarksopen = true,
		bookmarksopenlevel = 1,
		bookmarksnumbered = true,
		pdftitle = {基于Heisenberg模型的Skyrmion研究},%TODO:title
		pdfauthor = {杜治纬,孙晓晨,陶润恺,王金银,曾祥东},
		colorlinks,
		linkcolor = {red!60!black},
		citecolor = {green!50!black},
		urlcolor = {blue!70!black}
	}

\usepackage[stable, perpage]{footmisc}
\usepackage{pifont}
	\renewcommand{\thefootnote}{\ding{\numexpr171+\value{footnote}}}

\usepackage[numbers, square, super]{natbib}
	\setlength{\bibsep}{0 pt}
	\renewcommand{\citenumfont}{\itshape}
	\renewcommand{\bibsection}{\section{参考文献}}
	\bibliographystyle{gbt-7714-2015-numerical}

\usepackage{amsmath}
\usepackage{mathtools}
\usepackage[adobe-garamond]{mathdesign}
\usepackage{bm}

\usepackage[no-math]{fontspec}
	\setmainfont[Ligatures = TeX, BoldFont = AGaramondPro-Semibold.otf]{AGaramondPro-Regular.otf}

\usepackage[UTF8, heading = true, 10pt]{ctex}
\usepackage{xeCJK}
	\xeCJKsetup{
		AutoFakeBold = false,
		AutoFakeSlant = false
	}
	\setCJKmainfont[BoldFont = 华文中宋, ItalicFont = 华文楷体, Mapping = fullwidth-stop]{宋体}
	\setCJKfamilyfont{中宋}[Mapping = fullwidth-stop]{华文中宋}
	\setCJKfamilyfont{楷体}[Mapping = fullwidth-stop]{华文楷体}
	\setCJKfamilyfont{仿宋}[Mapping = fullwidth-stop]{华文仿宋}
	\setCJKfamilyfont{黑体}[BoldFont = 黑体, Mapping = fullwidth-stop]{黑体}
	
	\newfontfamily \courier {Courier New}
	\newfontfamily \verdana {Verdana}
	\newfontfamily \helvetica {HelveticaNeueLTPro-Roman.otf}[BoldFont = HelveticaNeueLTPro-Roman.otf]
	
	\newcommand{\myHeavy}{\helvetica \CJKfamily{黑体}}
	\renewcommand{\sffamily}{\verdana}
	\renewcommand{\ttfamily}{\courier}
%	\setcounter{secnumdepth}{4}
	\pagestyle{plain}
	\ctexset{
		%abstractname = {中},%{\myHeavy \normalsize 摘\phantom{空}要},
		appendix = {
			name = {附录}
		},
		section = {
			format = {\myHeavy \Large \centering},
			name = {,、\hspace{-1 em}},
			number = \chinese{section},
		},
		subsection = {
			format = {\myHeavy \large},
%			name = {,、\hspace{-1 em}},
%			numbering = true,
%			number = \chinese{subsection},
		}
	}
	\renewcommand{\abstractname}{\myHeavy \large 摘 \quad 要}

\usepackage{enumitem}

\usepackage{multicol}
\setlength{\columnsep}{20 pt}

\usepackage{listings}
\lstset{
	language = C++,
	basicstyle = \ttfamily \footnotesize,
	breaklines = true,
	tabsize = 4,
	numbers = left,
	numberstyle = \sffamily \itshape \tiny \color{gray},
	numbersep = 10 pt,
	showstringspaces = false,
	keywordstyle = \bfseries \color[rgb]{0,0,1},
	commentstyle = \color[rgb]{0,0.5,0},
	stringstyle = \color[rgb]{0.64,0.082,0.082},
	morecomment = [l][\color{violet}]{\#}
}

\usepackage{array}
	\newcolumntype{M}{>{$}c<{$}} %数学模式,居中
\usepackage{tabularx}
	\newcolumntype{Y}{>{\centering\arraybackslash}X} %定宽居中
\usepackage{booktabs} %三线表

\usepackage{graphicx}

\usepackage{float} %浮动体H选项
\usepackage{caption}
	\DeclareCaptionFont{myFigureCaptionFont}{\small}
	\DeclareCaptionFont{myTableCaptionFont}{\myHeavy \small}
	\captionsetup[figure]{
		font = myFigureCaptionFont,
		labelsep = quad,
		skip = 10 pt,
		position = bottom,
%		justification = centerfirst
		format = hang
	}
	\captionsetup[table]{
		font = myTableCaptionFont,
		labelsep = quad,
		skip = 10 pt,
		position = top
	}
\usepackage[labelformat=simple]{subcaption} %并排图表
\makeatletter
\renewcommand\p@subfigure{\thefigure-}
\renewcommand\thesubfigure{(\alph{subfigure})} %Subfig reference format: x-(x)
\makeatother

\input{Symbols.tex}

\newcommand{\emphA}[1]{{\myHeavy #1}}
\newcommand{\emphB}[1]{{\itshape #1}}
\newcommand{\figCaption}[2][]{\captionof{figure}{#2 \\ \emphB{#1}}}

\newenvironment{myAbstract}
	{\begin{abstract} \normalsize}
	{\end{abstract}}

\newenvironment{myFigure}
	{\par\medskip\noindent\minipage{\linewidth}\centering}
	{\endminipage\par\medskip}

\title{
	\vspace{-2 cm} \LARGE \bfseries
	基于Heisenberg模型的Skyrmion研究
}
\author{
	\CJKfamily{楷体}
	杜治纬,孙晓晨,陶润恺,王金银,曾祥东
}

\date{
	\CJKfamily{楷体}
	\today
}

\begin{document}
	\maketitle
	
	\begin{myAbstract}
		我们
	\end{myAbstract}
	
	\begin{multicols}{2}
	\section{引言}
		Skyrmion最初由Skyrme在1962年提出 ,用于刻画描述重子的行为。它是一种具有特定拓扑结构的准粒子,并被实验观察所证实。\cite{yu2010real}
		
		在磁性物质中,如果各电子自旋方向满足Skyrmion解,那就会形成一种稳定且具有手征性和拓扑性的磁结构。在二维磁性系统中,可以形成的一种Skyrmion呈涡旋形,自旋方向由中心向外周依次偏转,中心位置自旋向下,到最外周处自旋统一向上,如图 \ref{FIG_Skyrmions_WIKI} 所示。它的尺度是纳米数量级。由于自身的拓扑性质,Skyrmion能在外界扰动下保持稳定。%TODO:图,到底是哪一种
		
		\begin{myFigure}
			\includegraphics[width = 5 cm]{"Figures/FIG_Skyrmions_WIKI.png"}
			\captionof{figure}{二维Skyrmion的示意图\\图 (a)}
			\label{FIG_Skyrmions_WIKI}
		\end{myFigure}
		
		微观尺度上,Skyrmion的形成源于电子间的D-M相互作用。\cite{dzyaloshinsky1958thermodynamic,moriya1960anisotropic,wiki:MagneticSkyrmion} 它是由于界面附近空间对称性破坏而产生的,对应的Hamilton量可以表示为
		\begin{equation}
			H_\text{DM} = \vecb{D}_{12} \cdot (\vecb{s}_1 \times \vecb{s}_2) \comma
		\end{equation}
		其中的 $\vecb{s}_1$、$\vecb{s}_2$ 分别表示两个电子的自旋矢量,而 $\vecb{D}_{12}$ 是一个常矢量,由界面材料和晶体结构决定。
	\section{Heisenberg模型}
		\subsection{介绍}
			(经典)Heisenberg模型是开展Skyrmion等一系列研究的基础,它是 $n$-矢量模型在 $n = 3$ 时的情况。取一个 $d$ 维点阵,每个点阵点均是具有单位长度的三维自旋矢量
			\begin{equation}
				\vecb{s}_i \in \mathbb{R}^3, \quad \abs{\vecb{s}_i} = 1 \fullstop
			\end{equation}
			若系统的Hamilton量定义为
			\begin{equation} \label{EQ_DEF_OF_HEISENBERG_MODEL}
				\scH = -\sum_{\mean{i,j}} J_{i\!j} \vecb{s}_i \cdot \vecb{s}_j \comma
			\end{equation}
			则该自旋系统就是Heisenberg模型。\cite{joyce1967classical,wiki:ClassicalHeisenbergModel} 式~\eqref{EQ_DEF_OF_HEISENBERG_MODEL} 中的
			\begin{equation}
				J_{i\!j} =
				\begin{cases}
					J, & \text{若 $i$、$j$ 为近邻} \semicomma \\
					0, & \text{其他情况} \comma
				\end{cases}
			\end{equation}
			在Hamilton量中添加其他的相互作用,就可以得到各种外场下的情况。
			
		\subsection{算法}
			我们研究的是二维Heisenberg模型,并且不考虑外场。利用Metropolis算法(一种Monte Carlo算法),可以对该模型进行模拟。该算法的思路如下:
			\begin{enumerate}[itemsep = 0 pt]
				\item 给出自旋矢量在平面点阵中的一个随机分布。
				\item 选择一个自旋 $\vecb{s}_i$。
				\item 随机生成一个自旋矢量 $\vecb{s}'_i$,计算从 $\vecb{s}_i$ 翻转到 $\vecb{s}'_i$ 所带来的的能量变化 $\incr E$,从而得到翻转概率
					\begin{equation}
						P =
						\begin{cases}
							\exp \left(-\dfrac{\incr E}{\kB T}\right), &\incr E > 0 \semicomma \\
							1, &\incr E \leqslant 0 \comma
						\end{cases}
					\end{equation}
				\item 生成一个 $[0,1)$ 之间的随机数。若该随机数小于翻转概率 $P$,则将 $\vecb{s}_i$ 翻转为 $\vecb{s}'_i$。
				\item 重复进行步骤 $2\sim4$。我们采取的方式是遍历整个点阵,将它记为一个Monte Carlo步(MCS)。进行若干MCS后,体系就可以达到稳态。
				\item 在最后的若干MCS中计算各物理量,并取其平均值。\cite{周琼2010蒙特卡洛方法在磁性系统中的应用}%TODO:为什么
			\end{enumerate}
			
			在我们的模拟中,点阵大小取为 $36 \times 36$。整个过程包含1000个MCS,并在最后50步中计算各物理量。另外,所有参数均采用约化单位。其中,取 $\kB=1$,$J=1$。
			
			下面给出各物理量的计算公式。总能量
			\begin{equation}%TODO:系数
				E = \sum_{i} \ve_i \comma
			\end{equation}
			这里的 $\ve_i$ 即单个格点的Hamilton量。总磁矩
			\begin{equation}
				\vecb{M} = \sum_{i} \vecb{s}_i \fullstop
			\end{equation}
			比热
			\begin{equation} \label{EQ_HEAT_CAPACITY}
				C = \frac{\mean{E^2} - \mean{E}^2}{\kB T^2}
				= \frac{1}{\kB T^2} \left[ \frac{\sum_{i} \ve_i^2}{L^2} - \left( \frac{\sum_{i} \ve_i}{L^2} \right)^2 \right] \semicomma
			\end{equation}
			磁化率
			\begin{equation} \label{EQ_MAGNETIC_SUSCEPTIBILITY}
				\c = \frac{\mean{\vecb{M}^2} - \mean{\vecb{M}}^2}{\kB T}
				= \frac{1}{\kB T} \left[ \frac{\sum_{i} \vecb{s}_i^2}{L^2} - \left( \frac{\sum_{i} \vecb{s}_i}{L^2} \right)^2 \right] \fullstop
			\end{equation}
			以上两式中的 $L$ 为点阵边长。实际模拟中,各常数均忽略不计。
			
		\subsection{结果与分析}
			温度 $T$ 从0取到4,测量能量 $E$、总磁矩 $\vecb{M}$、比热 $C$ 和磁化率 $\c$ 随温度的变化。
%			\begin{figure}[h]
%				\centering
%				\begin{subfigure}[h]{0.47 \textwidth}
%					\includegraphics[width = 7cm]{"Figures/FIG_HEISENBERG_E_VS_T"}
%					\subcaption{能量}
%				\end{subfigure}
%				\begin{subfigure}[h]{0.47 \textwidth}
%					\includegraphics[width = 7cm]{"Figures/FIG_HEISENBERG_M_VS_T"}
%					\subcaption{总磁矩的大小}
%				\end{subfigure}
%%				\begin{subfigure}[h]{0.47 \textwidth}
%%					\includegraphics[width = 7cm]{"Figures/FIG_HEISENBERG_M_VS_T"}
%%					\subcaption{热容}
%%				\end{subfigure}
%%				\begin{subfigure}[h]{0.47 \textwidth}
%%					\includegraphics[width = 7cm]{"Figures/FIG_HEISENBERG_M_VS_T"}
%%					\subcaption{磁化率}
%%				\end{subfigure}
%				\caption{各物理量随温度的变化}
%			\end{figure}
			\begin{myFigure}
				\includegraphics[width = \textwidth]{Figures/FIG_HEISENBERG_E_M_VS_T}
				\figCaption[灰色曲线表示原始数据,蓝色曲线表示平滑后的结果;数据已进行单位化,即将原始值除以格点数。]{能量、总磁矩大小随温度的变化}
			\end{myFigure}
			
			由于式 \eqref{EQ_HEAT_CAPACITY} 和 \eqref{EQ_MAGNETIC_SUSCEPTIBILITY} 在 $T \approach 0$ 时是发散的,因此热容与磁化率利用能量、总磁矩曲线的导数来计算,如图 
			\ref{FIG_HEISENBERG_C_X_VS_T} 中的蓝色曲线所示。
			\begin{myFigure}
				\includegraphics[width = \textwidth]{Figures/FIG_HEISENBERG_C_X_VS_T}
				\figCaption[灰色曲线表示原始数据,蓝色曲线表示利用能量、总磁矩曲线的导数(差分)计算出的结果。]{热容、磁化率随温度的变化}
				\label{FIG_HEISENBERG_C_X_VS_T}
			\end{myFigure}
			
			要写分析!!!
	\section{Skyrmion的模拟}
	\section{简要的数学}
	\section{总结}
	\section{致谢}
		首先感谢陈焱老师在全过程的帮助,尤其是对我们选题的指导。其次要感谢吴义政老师,他从实验的角度提出了一些很有价值的观点。还要感谢左光宏老师,他帮助我们实现了并行计算。最后要感谢张宇学长、秦金谷学姐和李超然学长等人的支持与帮助,他们的科研经历给我们的课题带来了很多启发。
		
%	\section{参考文献}
		\nocite{LippmanLajoieMooEtAl2013}
		{
			\flushleft \small
			\bibliography{Reference}
		}
	\end{multicols}
	\appendix
	\section{评分}
	\section{代码实现}
		Heisenberg模型与产生Skyrmion的系统,差别仅在于Hamilton量,因此代码是相同的。实际模拟中,为了加快运行速度,Heisenberg模型中其他相互作用项的代码未被编译。
		
		矢量类(\texttt{MyVector})、计时类(\texttt{MyTiming})和若干简单函数的实现代码没有给出。
		\linespread{1}
%		\subsection{Head.h}
%		\lstinputlisting{../HeisenbergModel/Head.h}
%		\subsection{MyLattice.h}
%		\lstinputlisting{../HeisenbergModel/MyLattice.h}
%		\subsection{MyLattice.cpp}
%		\lstinputlisting{../HeisenbergModel/MyLattice.cpp}
%		\subsection{Physics.h}
%		\lstinputlisting{../HeisenbergModel/Physics.h}
%		\subsection{Physics.cpp}
%		\lstinputlisting{../HeisenbergModel/Physics.cpp}
%		\subsection{SingleSimulation.h}
%		\lstinputlisting{../HeisenbergModel/SingleSimulation.h}
%		\subsection{SingleSimulation.cpp}
%		\lstinputlisting{../HeisenbergModel/SingleSimulation.cpp}
		\subsection{Main.cpp}
		\lstinputlisting{../HeisenbergModel/HeisenbergModel_Main.cpp}
\end{document}