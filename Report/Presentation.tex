\documentclass[UTF8, aspectratio = 43]{beamer}
	\usetheme{CambridgeUS}
	\setbeamercolor{caption name}{fg = darkred}
	\setbeamersize{text margin left = 1cm, text margin right = 1cm}
	\setbeamertemplate{caption}[numbered]
	\setbeamertemplate{bibliography item}{\insertbiblabel}
	\setbeamerfont{institute}{size = \small}
	\setbeamerfont{date}{size = \small}
	\setbeamerfont{footnote}{size = \tiny}
	\logo{\includegraphics[width = 0.8cm]{Figures/Fudan_University_Emblem.pdf}}

	\hypersetup{pdfpagemode=FullScreen}
	
	\bibliographystyle{gbt-7714-2015-numerical}

\usepackage{cmbright}
\SetSymbolFont{largesymbols}{normal}{OMX}{iwona}{m}{n}
\SetSymbolFont{symbols}{normal}{OMS}{iwona}{m}{n}
\usepackage{bm}

\makeatletter
\g@addto@macro\normalsize{%
	\setlength\abovedisplayskip{40pt}
	\setlength\belowdisplayskip{40pt}
	\setlength\abovedisplayshortskip{40pt}
	\setlength\belowdisplayshortskip{40pt}
}
\makeatother

%\usepackage[no-math]{fontspec}
%	\setsansfont{Calibri}
\usepackage{ctex}
\usepackage{xeCJK}
	\xeCJKsetup{
		AutoFakeBold = false,
		AutoFakeSlant = false
	}
	\setCJKsansfont[BoldFont = 思源黑体 CN Medium, Mapping = fullwidth-stop]{思源黑体 CN Normal}
\usepackage{multicol}
%\usepackage{tabularx}
%\newcolumntype{Y}{>{\centering\arraybackslash}X}
%\usepackage{booktabs}

\usepackage{siunitx}
	\sisetup{
		number-math-rm = \ensuremath,
		inter-unit-product = \ensuremath{{}\cdot{}},
		group-digits = true,
		group-minimum-digits = 5,
		group-separator = \text{~},
		separate-uncertainty = true,
		range-phrase = \text{~}$\sim$\text{~},
		range-units = single
	}

\usepackage{graphicx}
\usepackage[labelformat = simple]{subcaption}
	\renewcommand\thesubfigure{(\alph{subfigure})}

%大写正体字母
\newcommand{\rmY}{\mathrm{Y}}

%大写手写体字母
\newcommand{\scF}{\mathcal{F}}
\newcommand{\scH}{\mathcal{H}}
\newcommand{\scI}{\mathcal{I}}
\newcommand{\scJ}{\mathcal{J}}
\newcommand{\scL}{\mathcal{L}}
\newcommand{\scP}{\mathcal{P}}
\newcommand{\scR}{\mathcal{R}}
\newcommand{\scS}{\mathcal{S}}

%小写希腊字母
\renewcommand{\a}{\alpha}
\renewcommand{\b}{\beta}
\newcommand{\g}{\gamma}
\renewcommand{\d}{\delta}
\newcommand{\e}{\epsilon}
\newcommand{\ve}{\varepsilon}
\newcommand{\z}{\zeta}
\newcommand{\h}{\eta}
\newcommand{\q}{\theta}
\renewcommand{\th}{\theta}
\newcommand{\vq}{\vartheta}
\newcommand{\vth}{\vartheta}
\renewcommand{\i}{\iota}
\renewcommand{\k}{\varkappa} %这两个是相反的
\newcommand{\vk}{\kappa} %这两个是相反的
\renewcommand{\l}{\lambda}
\renewcommand{\m}{\mu}
\newcommand{\n}{\nu}
\newcommand{\x}{\xi}
%\renewcommand{\p}{\pi} %\p 已定义为正体π
\newcommand{\vp}{\varpi}
\renewcommand{\r}{\rho}
\newcommand{\vr}{\varrho}
\newcommand{\s}{\sigma}
\newcommand{\vs}{\varsigma}
\renewcommand{\t}{\tau}
\renewcommand{\u}{\upsilon}
\newcommand{\f}{\phi}
\renewcommand{\j}{\varphi}
\newcommand{\vf}{\varphi}
\newcommand{\irchi}[2]{\raisebox{\depth}{$#1\chi$}}
\renewcommand{\c}{{\mathpalette\irchi\relax}}
\newcommand{\y}{\psi}
\renewcommand{\o}{\omega}
\newcommand{\w}{\omega}

%大写希腊字母
\renewcommand{\G}{\varGamma}
\newcommand{\D}{\varDelta}
\newcommand{\Q}{\varTheta}
\newcommand{\Th}{\varTheta}
\renewcommand{\L}{\varLambda}
\newcommand{\X}{\varXi}
\renewcommand{\P}{\varPi}
\let \secSymbol = \S
\renewcommand{\S}{\varSigma}
\renewcommand{\U}{\varUpsilon}
\newcommand{\F}{\varPhi}
\newcommand{\Y}{\varPsi}
\renewcommand{\O}{\varOmega}
\newcommand{\W}{\varOmega}

%常数
\newcommand{\p}{\piup} %圆周率
\newcommand{\ee}{\mathrm{e}} %自然对数基
\newcommand{\ii}{\mathrm{i}} %虚数单位
\newcommand{\hb}{\hbar} %约化Planck常数
\newcommand{\kB}{k_{\mathrm{B}}} %Boltzmann常数
\newcommand{\pfs}{\e_0} %真空介电常数 Permittivity of Free Space

%数学修饰符号
\newcommand{\vecb}[1]{\bm{#1}} %矢量用斜体,根据ISO 80000-2/2009
\renewcommand{\bar}[1]{\overline{#1}} %上横线
\newcommand{\tl}[1]{\widetilde{#1}} %弯号

%数学函数
\newcommand{\abs}[1]{\left\lvert #1 \right\rvert} %绝对值
\newcommand{\mean}[1]{\left< #1 \right>} %统计平均

%算子、算符
%已经调整好前后间距
\DeclareMathOperator{\incr}{\Delta\!}
\DeclareMathOperator{\dd}{\mathrm{d}\!} %微分 Differential D
\DeclareMathOperator{\db}{{\mkern2.5mu\mathchar'26\mkern-11mu\mathrm{d}}\!} %不完全微分 Inexact Differential D; Only for "Adobe Garamond" font!
\DeclareMathOperator{\pd}{\partial\!} %偏微分 Partial D
\DeclareMathOperator{\vd}{\deltaup\!} %变分 Variational D
\DeclareMathOperator{\laplac}{\nabla^2\!} %Laplace算子

%复合数学符号
\newcommand{\infint}{\int_{-\infty}^{+\infty}} %上下限无穷的积分
\newcommand{\iTonSum}{\sum_{i = 1}^{n}} %求和 i=1..n
\newcommand{\iToNSum}{\sum_{i = 1}^{N}} %求和 i=1..N
\newcommand{\nEnum}[2]{#1 = 1, \, 2, \, \dots, \, #2} %如x=1,2,...,n
%\newcommand{\coordinateLabel}[2]{#1\text{-}#2} %如p-V坐标
\newcommand{\myPartial}[3]{\left( \frac{\pd #1}{\pd #2} \right)_{#3}} %带括号的偏微分
\newcommand{\myPartialDisplay}[3]{\left( \dfrac{\pd #1}{\pd #2} \right)_{#3}}

%重定义符号
\newcommand{\approach}{\rightarrow} %趋向于
\newcommand{\eqdef}{\equiv} %定义等号(三线)
\newcommand{\dotTimes}{\cdot} %点乘

%公式用标点、文字
\newcommand{\comma}{\text{,}}
\newcommand{\fullstop}{\text{.}}
\newcommand{\semicomma}{\text{;}}
\newcommand{\const}{\text{const.}}

%正文用符号
%\catcode`\。 = \active
%\newcommand{。}{.} %句号用句点代替,已用xeCJK的Mapping代替
%\newcommand{\mySec}{{\fontspec{Times New Roman} \S \,}} %TODO:20160302 冲突;章节符号,Garamond 字体不好看
\renewcommand{\emph}[1]{ \textcolor{darkred} {#1} }
\newcommand{\keyword}[1]{\emph{#1}\quad}

\title{基于Heisenberg模型的skyrmion研究}
%\subtitle{SPR Based on Spectrometer}
\author[曾祥东]{杜治纬,孙晓晨,陶润恺,王金银,曾祥东}
\institute{复旦大学物理系}
\date{2016年6月15日}

\begin{document}
	\begin{frame}
		\titlepage
	\end{frame}
	
	\section{Introduction}
		\begin{frame}{Heisenberg Model}
			\begin{columns}
				\begin{column}{0.46 \textwidth}
					\begin{itemize}
						\item Why? \pause
						\item Hamilton量:
						\begin{equation*}
							\scH = -\sum_{\mean{i,\,j}} J_{ij} \,\textcolor{darkred}{\vecb{s}_i \!\cdot \!\vecb{s}_i} \, - H_z\sum_i s_i^{\,z}
						\end{equation*}
						其中 $\vecb{s}_i \in \mathbb{R}^3, \quad \abs{\vecb{s}_i\,} = 1$
						并且满足
						\begin{equation*}
							J_{ij} =
							\begin{cases}
							J, & \text{若 $i$、$j$ 近邻} \semicomma \\
							0, & \text{其他情况}
							\end{cases}
						\end{equation*}
					\end{itemize}
				\end{column} \pause
				\begin{column}{0.56 \textwidth}
					\begin{itemize}
						\small
						\item 给出平面点阵的一个随机分布
						\item 选择一个自旋 $\vecb{s}_i$
						\item 随机生成另一个自旋矢量 $\vecb{s}_i^{\,\prime}$,计算翻转带来的的 $\Delta E$,得到\emph{翻转概率} $P$
						\item 生成一个 $[0,1)$ 之间的随机数,若小于 $P$,则将 $\vecb{s}_i$ 翻转为 $\vecb{s}_i^{\,\prime}$
						\item 重复步骤 $2\sim4$。遍历整个点阵,将它记为一个\emph{MCS}。进行若干MCS后,认为体系达到稳态
						\item 在最后的若干MCS中作系综平均,计算各物理量
					\end{itemize}
				\end{column}
			\end{columns}
		\end{frame}
		
		\begin{frame}{Skyrmion \& DM Interaction}
			\small
			\begin{columns}
				\begin{column}{0.45\textwidth}
					\begin{itemize}
						\item 一种拓扑激发态(T.~Skyrme, 1962)
						\item 元激发 or 准粒子
					\end{itemize}
					\begin{figure}
						\vspace{-0.4cm}
						\includegraphics[width = 0.8\textwidth] {Figures/FIG_Skyrmions_WIKI.png}
						\caption{二维skyrmion的示意图\footnotemark}
					\end{figure}
				\end{column} \pause
				\begin{column}{0.55\textwidth}
					\begin{itemize}
						\item DM相互作用:空间对称性破缺
						\item Hamilton量:
						\vspace{-0.2cm}
						\begin{equation*}
							\scH_\text{DM} = \vecb{D}_{12} \cdot (\vecb{s}_1 \times \vecb{s}_2)
						\end{equation*} \pause
						\vspace{-0.8cm}
						\item 总Hamilton量:
						\vspace{-0.2cm}
						\begin{equation*}
							\scH_\text{total} = \text{J} + \text{D} + \text{H}
						\end{equation*} \pause
						\vspace{-0.8cm}
						\item 三种相互作用的竞争
						\item 四种不同的相(低温)
						\begin{itemize}
							\item Helical, conical, skyrmion, FM
						\end{itemize} \pause
						\item Skyrmion密度(SkD)
						\vspace{-0.2cm}
						\begin{equation*}
							\phi =\frac{1}{4\pi} \vecb{s} \cdot \left(\frac{\pd \vecb{s}}{\pd x}\times\frac{\pd \vecb{s}}{\pd y}\right)
						\end{equation*}
					\end{itemize}
				\end{column}
				\footnotetext{图片来源:\url{https://en.wikipedia.org/wiki/Magnetic_skyrmion}}
			\end{columns}
		\end{frame}
		
	\section{Our Results}
		\begin{frame}{Heisenberg Model}
			\begin{columns}
				\begin{column}{0.47\textwidth}
					\begin{figure}
						\includegraphics[width = \textwidth]{Figures/FIG_HEISENBERG_E_M_VS_T_LATTICE_10}
						\caption{$E$、$\abs{\vecb{M}\,}$ 随温度的变化。点阵大小为 $10\times10$,步数为 \num{100000}。}
					\end{figure} \pause
					\vspace{-1cm}
					\begin{figure}
						\includegraphics[width = \textwidth]{Figures/FIG_HEISENBERG_C_X_VS_T_LATTICE_10}
						\caption{$C$、$\c$ 随温度的变化。参数同上。}
					\end{figure}
				\end{column} \pause
				\begin{column}{0.53\textwidth}
					\begin{figure}
						\includegraphics[width = 0.78 \textwidth]{Figures/FIG_HEISENBERG_STEP_FM}
						\caption{铁磁相随模拟步数的演化}
					\end{figure} \pause
					\vspace{-1cm}
					\begin{figure}
						\includegraphics[width = 0.78 \textwidth]{Figures/FIG_HEISENBERG_STEP_AFM}
						\caption{反铁磁相随模拟步数的演化}
					\end{figure}
				\end{column}
			\end{columns}
		\end{frame}
		
		\begin{frame}{Skyrmion (I)}
			\begin{columns}
				\begin{column}{0.5\textwidth}
					\begin{figure}
						\includegraphics[width = \textwidth]{Figures/FIG_PHASE_CURVE}
						\caption{Skyrmion密度和磁矩 $z$ 分量随外加磁场的变化}
					\end{figure} \pause
					\vspace{-1cm}
					\begin{figure}
						\includegraphics[width = 2.5 cm]{Figures/FIG_ORIGINAL_VECTOR_CONICAL}
						\caption{受skyrmion抑制的锥形相}
					\end{figure}
				\end{column} \pause
				\begin{column}{0.5\textwidth}
					\begin{figure}
						\includegraphics[width = \textwidth]{Figures/FIG_LATTICE_VECTOR}
						\caption{磁矩在 $xy$ 平面上的投影}
					\end{figure}
					\vspace{-1cm}
					\begin{figure}
						\includegraphics[width = \textwidth]{Figures/FIG_LATTICE_MATRICX}
						\caption{磁矩的 $z$ 分量}
					\end{figure}
					\vspace{-0.8cm}
					\begin{center}
						\tiny
						Conical:$J=1\comma H=0\comma D=2.5\comma T=0.2$
						
						Skyrmion:$J=1\comma H=2\comma D=2.5\comma T=0.1$
						
						FM:$J=1\comma H=4\comma D=2.5\comma T=0.1$
					\end{center}
				\end{column}
			\end{columns}
		\end{frame}
		
		\begin{frame}{Skyrmion (II) --- Vector Diagram}
			\begin{center}
				\begin{figure}
					\includegraphics[width = \textwidth]{Figures/FIG_VECTOR_3D}
					\caption{从左至右依次为锥形相、skyrmion相和铁磁相}
				\end{figure}
			\end{center}
		\end{frame}
		
		\begin{frame}{Skyrmion (III) --- Phase Diagram}
			\begin{columns}
				\begin{column}{0.6\textwidth}
					\begin{figure}
						\includegraphics[width = \textwidth]{Figures/FIG_PHASE_SD_AND_MZ}
						\caption{SkD与 $m_z$ 随外加磁场 $\vecb{H}$ 和温度 $T$ 的变化}
					\end{figure}
				\end{column} \pause
				\begin{column}{0.4\textwidth}
					\begin{figure}
						\includegraphics[width = 4cm]{Figures/FIG_PHASE}
						\caption{$T$-$H$ 相图}
					\end{figure}
				\end{column}
			\end{columns}
		\end{frame}
		
		\begin{frame}{Skyrmion (IV) --- Ising \& XY Limit}
			\begin{columns}
				\begin{column}{0.5\textwidth}
					\begin{figure}
						\includegraphics[width = 4cm]{Figures/FIG_LIMIT_ISING}
						\caption{趋向Ising极限的过程}
					\end{figure}
					\vspace{-1cm}
					\begin{figure}
						%						\includegraphics[width = \textwidth]{Figures/FIG_LIMIT_ISING_VECTOR}
						%						
						\includegraphics[width = \textwidth]{Figures/FIG_LIMIT_ISING_MATRIX}
						\caption{类Ising情形下的三种相}
					\end{figure}
				\end{column} \pause
				\begin{column}{0.5\textwidth}
					\begin{figure}
						\includegraphics[width = 4cm]{Figures/FIG_LIMIT_XY}
						\caption{趋向XY极限的过程}
					\end{figure}
					\vspace{-1cm}
					\begin{figure}
%						\includegraphics[width = \textwidth]{Figures/FIG_LIMIT_XY_VECTOR}
%						
						\includegraphics[width = \textwidth]{Figures/FIG_LIMIT_XY_MATRIX}
						\caption{类XY情形下的三种相}
					\end{figure}
				\end{column}
			\end{columns}
		\end{frame}
		
	\section{Afterwards}
		\begin{frame}{Unsolved Problems}
			\begin{itemize}
				\item 理论? \pause
				\item 低温下热容与磁化率的发散 \pause
				\item 稳定过程 \&“神奇的突跃”
			\end{itemize}
			\begin{figure}
				\includegraphics[width = 10cm]{Figures/FIG_DYNAMIC}
				\caption{平均能量、平均磁矩的演化曲线}
			\end{figure}
		\end{frame}
		
		\begin{frame}{Suggestions}
			\begin{itemize}
				\footnotesize
				\item \emph{第10组}
				
				各种相的稳定性与Monte Carlo步数之间的关系并不能明确,且证据不充分;
				
				高温部分,各种相在同一条件下混杂并存并不能明显体现;
				
				未能对能量与磁矩的现突变给出合理解释。
				
				\item \emph{第11组}
				
				从Hamilton量到直接提出算法有点突兀,可以适当加入一些公式推导;
				
				没有探讨大型点阵的模拟结果。
				
				\item \emph{第12组}
				
				$T$ 在趋于0时,由于涨落,所得热力学量会发散,此外通过差分得到的结果亦不满足热力学第三定律,可见在低温时模拟的可靠性是存疑的;
				
				未明确为何可以合理地使用 $36\times36$ 的点阵计算。
				
				\item \emph{第13组}
				
				对材料性质的研究较少;
				
				建议对各向异性指数 $\alpha$ 进行一些数学上的说明,并说明其对热力学量的影响。
				
				\item \emph{第14组}
				
				部分图片过小导致图中一些数据难以看清;
				
				代码部分不妨加入一些注释。
			\end{itemize}
		\end{frame}
		
		\section{Question Time}
		\begin{frame}
			\begin{center}
				\Huge
				\textcolor{darkgray}{Any questions?}
			\end{center}
		\end{frame}
		
%		\section{Acknowledgements}
%		\begin{frame}
%			\begin{center}
%				\Huge
%				谢谢大家!
%			\end{center}
%		\end{frame}
\end{document}
\documentclass{article}
\usepackage[T1]{fontenc}